\documentclass{beamer}
%\documentclass[handout]{beamer}
\usetheme{neo}
\usepackage{tikz}
\usetikzlibrary{arrows,backgrounds,decorations.markings,shapes.misc,spy}

% Gefunden auf: https://texample.net/tikz/examples/double-arrows/
\tikzstyle{vecArrow} = [thick, decoration={markings,mark=at position
    1 with {\arrow[semithick]{open triangle 60}}},
    double distance=1.4pt, shorten >= 5.5pt,
    preaction = {decorate},
    postaction = {draw,line width=1.4pt, white,shorten >= 4.5pt}]

% Siehe: https://tex.stackexchange.com/questions/123760/draw-crosses-in-tikz
\tikzset{cross/.style={cross out, draw=black, minimum size=2*(#1-\pgflinewidth), inner sep=0pt, outer sep=0pt, line width=2pt},
%default radius will be 1pt.
cross/.default={8pt}}

\usepackage[utf8]{inputenc} % UTF-8 Kodierung verwenden
\usepackage[ngerman]{babel} % Neue deutsche Sprache
\usepackage{xcolor}
\usepackage{textpos}
\usepackage[percent]{overpic}
\usepackage{soul}
\usepackage{pgfplots}
\usepackage{booktabs} % \midrule
\usepackage{makecell} % \makecell
\usepackage{tabulary}
\usepackage{chemformula}





\newcommand{\gerquot}[1]{\glqq#1\grqq}
\newcommand{\dashAndSpace}{\textendash \space}
\newcommand{\dashAndSpaceSeq}[1]{\dashAndSpace#1 \dashAndSpace}
\newcommand{\tikzScale}{0.75}
\newcommand{\massCharge}{$ m/z $ }
\newcommand{\xAxisUnit}{\massCharge}
\newcommand{\yAxisUnit}{$y$}
\newcommand{\yAxisHeight}{3}
\newcommand{\xAxisLength}{5}
\newcommand{\axisColorOffset}{0.15}
\newcommand{\highlightColor}{nDarkBlue!55!}


\title{De-Novo-Sequencing using Spectrum-Graphs, enabling Open Searches}
\author{Dominik Habermann}
\date{14. Juli 2023}
\institute{Ruhr Universität Bochum}



%%%%% %%%%% %%%%% %%%%% %%%%% %%%%% %%%%% %%%%%                %%%%% %%%%% %%%%% %%%%% %%%%% %%%%% %%%%% %%%%%
%%%%% %%%%% %%%%% %%%%% %%%%% %%%%% %%%%% %%%%% BEGIN document %%%%% %%%%% %%%%% %%%%% %%%%% %%%%% %%%%% %%%%%
%%%%% %%%%% %%%%% %%%%% %%%%% %%%%% %%%%% %%%%%                %%%%% %%%%% %%%%% %%%%% %%%%% %%%%% %%%%% %%%%%
\begin{document}
    \maketitle

    % * Praesentation fuer Bioinformatik Seminar *
    % - Dauer: 40 Min
    %
    % - Vortrag: 30 Min
    % - Diskussion / Fragen: 10 Min
    \begin{frame}{Gliederung}
        \tableofcontents
    \end{frame}

    %%%%% %%%%% %%%%% %%%%% %%%%% BEGIN Aminosäure Sequenzierung %%%%% %%%%% %%%%% %%%%% %%%%%
    \section{Aminosäure Sequenzierung}
    \begin{frame}{Hintergrund}
        \newcommand{\AARadius}{8pt}
        \newcommand{\AATikzPictureScale}{0.625}
        \newcommand{\AAMinipageSize}{0.31}
        \newcommand{\circleColor}{purple!85!}

        %%%%% %%%%% BEGIN AA Schema %%%%% %%%%% %%%%%
        \onslide<1->{
            \begin{minipage}[t]{\AAMinipageSize\textwidth}
                \centering
                \begin{tikzpicture}[scale=\AATikzPictureScale]
                    \node (0) at (1, 1.25) {};
                    \node (1) at (1, 0) {};
                    \node (2) at (2, -1) {};
                    \node (3) at (2, 0.25) {};
                    \node (4) at (2.75, 0) {};
                    \node (5) at (3.25, 0.5) {};
                    \node (6) at (2.5, 1) {};
                    \node (7) at (2, 2) {};

                    \foreach \i in {0,...,7}
                    {
                        \filldraw [shade, shading=radial, inner color=white, outer color=\circleColor] (\i) circle (\AARadius);
                    }
                    % Debug Kreis
                    % \filldraw [shade, shading=radial, inner color=white, outer color=black] (0, 0) circle (\AARadius);
                \end{tikzpicture}
                Aminosäure (AA)
            \end{minipage}
        }%
        %%%%% %%%%% END AA Schema %%%%% %%%%% %%%%%
        \hfill
        %%%%% %%%%% BEGIN Peptide Schema %%%%% %%%%% %%%%%
        \onslide<2->{
            \begin{minipage}[t]{\AAMinipageSize\textwidth}
                \centering
                \begin{tikzpicture}[scale=\AATikzPictureScale]
                    \node (0) at (0.75, 0.25) {};
                    \node (1) at (0.75, 1.25) {};
                    \node (2) at (2, 0.25) {};
                    \node (3) at (1.5, 0.75) {};
                    \node (4) at (2.75, 1) {};
                    \node (5) at (2, 1.75) {};
                    \node (6) at (1.5, -0.5) {};
                    \node (7) at (3.25, 2) {};
                    \node (8) at (2.75, 0.25) {};

                    \draw[ultra thick, -]  (0.center) to [bend left]  (1.center);
                    \draw[ultra thick, -]  (1.center) to [bend left]  (3.center);
                    \draw[ultra thick, -]  (3.center) to [bend left]  (5.center);
                    \draw[ultra thick, -]  (6.center) to [bend left]  (2.center);
                    \draw[ultra thick, -]  (2.center) to [bend left]  (8.center);
                    \draw[ultra thick, -]  (7.center) to [bend left]  (4.center);
                    \draw[ultra thick, -]  (4.center) to [bend left]  (8.center);

                    \foreach \i in {0,...,8}
                    {
                        \filldraw [shade, shading=radial, inner color=white, outer color=\circleColor] (\i) circle (\AARadius);
                    }
                    % Debug Kreis
                    % \filldraw [shade, shading=radial, inner color=white, outer color=black] (0, 0) circle (\AARadius);
                \end{tikzpicture}
                \newline % Ein manuelles \newline ist hier erforderlich !
                Peptid
            \end{minipage}
        }%
        %%%%% %%%%% END Peptide Schema %%%%% %%%%% %%%%%
        \hfill
        %%%%% %%%%% BEGIN Protein Schema %%%%% %%%%% %%%%%
        \onslide<3->{
            \begin{minipage}[t]{\AAMinipageSize\textwidth}
                \centering
                \begin{tikzpicture}[scale=\AATikzPictureScale]
                    \node (0) at (1.25, -0.5) {};
                    \node (1) at (0.5, 1) {};
                    \node (2) at (1.5, 1.25) {};
                    \node (3) at (1.25, 0.75) {};
                    \node (4) at (2, -0.5) {};
                    \node (5) at (3, 0.75) {};
                    \node (6) at (2, 0.75) {};
                    \node (7) at (3.25, 2) {};
                    \node (8) at (2.75, 1.25) {};
                    \node (9) at (2.25, 1.5) {};
                    \node (10) at (2, 2) {};
                    \node (11) at (1, 1.5) {};
                    \node (12) at (0.75, 2.25) {};
                    \node (13) at (1.5, 3) {};
                    \node (14) at (1.5, 0) {};
                    \node (15) at (2.5, 2.5) {};
                    \node (16) at (2.5, 0.25) {};
                    \node (17) at (1.5, 2.25) {};

                    \draw[ultra thick, -] (0.center) to [bend left]  (1.center);
                    \draw[ultra thick, -] (1.center) to [bend left]  (3.center);
                    \draw[ultra thick, -] (3.center) to [bend left]  (14.center);
                    \draw[ultra thick, -] (14.center) to [bend left]  (4.center);
                    \draw[ultra thick, -] (6.center) to [bend left]  (5.center);
                    \draw[ultra thick, -] (5.center) to [bend left]  (2.center);
                    \draw[ultra thick, -] (2.center) to [bend left]  (9.center);
                    \draw[ultra thick, -] (9.center) to [bend left]  (8.center);
                    \draw[ultra thick, -] (8.center) to [bend left]  (7.center);
                    \draw[ultra thick, -] (10.center) to [bend left]  (11.center);
                    \draw[ultra thick, -] (11.center) to [bend left]  (12.center);
                    \draw[ultra thick, -] (12.center) to [bend left]  (13.center);
                    \draw[ultra thick, -] (4.center) to [bend left]  (16.center);
                    \draw[ultra thick, -] (16.center) to [bend left]  (6.center);
                    \draw[ultra thick, -] (15.center) to [bend left]  (17.center);
                    \draw[ultra thick, -] (17.center) to [bend left]  (10.center);
                    \draw[ultra thick, -] (15.center) to [bend left]  (7.center);

                    \foreach \i in {0,...,17}
                    {
                        \filldraw [shade, shading=radial, inner color=white, outer color=\circleColor] (\i) circle (\AARadius);
                    }
                    % Debug Kreis
                    % \filldraw [shade, shading=radial, inner color=white, outer color=black] (0, 0) circle (\AARadius);
                \end{tikzpicture}
                Protein
            \end{minipage}
        }%
        %%%%% %%%%% END Protein Schema %%%%% %%%%% %%%%%

        \onslide<4->{
            \vspace*{0.75cm}
            \begin{itemize}
                \item Peptid $\widehat{=}$ Kurze Ketten an AA
                \item Protein $\widehat{=}$ Verkettung von Peptiden
            \end{itemize}
        }%
    \end{frame}


    \begin{frame}{AA Sequenzierung}
        \begin{itemize}
            \item<1-> Ziel: Bestimmung der \textcolor{\highlightColor}{AA-Sequenz} von Peptiden
            \vspace*{0.5cm}
            \item<2-> Warum ist die AA-Sequenz relevant?
        \end{itemize}
    \end{frame}


    % Am Ende dieser Folie u.U. zwei Aminosäuren, die direkt erkennbar eine deutlich unterschiedliche Struktur haben, mithilfe des Molekülbaukastens zeigen. Frei nach dem Motto: "Man erkennt sofort, dass wenn A durch B ausgetauscht wird, eine nennenswerte Änderung stattfindet."
    \begin{frame}{AA Sequenzierung \dashAndSpace Relevanz}
        \begin{itemize}
            \item<1-> Reihenfolge der AA hat unter anderem Einfluss auf:
            \begin{itemize}
                \item<2-> 3D Aufbau eines Proteins
                \item<2-> Funktionsweise
                \item<2-> Fähigkeiten
                \item<2-> Notwendigen Umgebungsbedingungen (Temperatur, pH-Wert, etc.)
                \end{itemize}
            \item<3-> $\Rightarrow$ AA-Sequenz ist von \textcolor{\highlightColor}{wesentlicher Bedeutung}
        \end{itemize}
    \end{frame}


    \begin{frame}{AA Sequenzierung \dashAndSpace Relevanz}
        \begin{itemize}
            \item<1-> Biomedizinische Relevanz:
            \begin{itemize}
                \item<2-> Katalogisierung von Proteinen
                \item<2-> Analyse von Enzymen
                \item<2-> Toxikologie von Proteinen
            \end{itemize}
            \vspace*{0.5cm}
            \item<3-> Zuverlässige Sequenzierung möglich?
        \end{itemize}
    \end{frame}


    \begin{frame}{AA Sequenzierung \dashAndSpace Suchraum}
        \newcommand{\numAA}{20}
        \onslide<1->{
            \begin{itemize}
                \item<1-> \textcolor{\highlightColor}{\numAA} relevante AA
                \item<1-> Weitestgehend beliebig kombinierbar
                \item<1-> Bereits bei wenigen AA: \textcolor{\highlightColor}{Kaum händelbarer Suchraum}
            \end{itemize}
        }%
        \onslide<2->{
            \centering
            \begin{tikzpicture}[scale=0.65]
                \begin{axis} [axis lines=center,xlabel=$x$,
                    ylabel={$f(x)=\numAA^x$}]
                    \addplot [domain=0:5, smooth, thick] { \numAA^x };
                \end{axis}
            \end{tikzpicture}
        }%
        \onslide<3->{
            \begin{itemize}
                \item<3-> Zum Vergleich: Proteine bis zu mehreren \emph{zehntausend} AA
            \end{itemize}
        }%
    \end{frame}


    \begin{frame}{AA Sequenzierung \dashAndSpace Hilfsmittel Massenspektrometrie}
        \begin{itemize}
            \item<1-> $\Rightarrow$ Intelligentes Sequenzierungsverfahren notwendig
        \end{itemize}
        \begin{itemize}
            \item<2-> Hilfsmittel: \textcolor{\highlightColor}{Massenspektrometrie (MS)}
            \item<2-> MS kann chemische Strukturen bestimmen
            \item<2-> Rückschluss auf die AA-Sequenz möglich
        \end{itemize}
    \end{frame}


    \begin{frame}{AA Sequenzierung \dashAndSpace Hilfsmittel MS}
        \newcommand{\movLineColor}{green}
            \begin{itemize}
                \item<1-> MS erzeugt \textcolor{\highlightColor}{MS-Spektren}
                \item<2-> Beispiel: vereinfachte Darstellung von realen Messwerten
            \end{itemize}
        \vspace*{0.2cm}

        \onslide<2->{
            \centering
            % Siehe: https://imathworks.com/tex/tex-latex-tikz-why-is-dimension-too-large/
            % Bei Problemen mit "dimension is too large" errors
            \begin{tikzpicture}[scale=0.02, x=0.550cm, y=0.0045cm, baseline=(current bounding box.center)]
                % JA es muessen die x-Werte (m/z) Werte auf den Peaks geschrieben werden !
                %\node (0) at (149.04489, 15549.466) [above] {\tiny 149.04};
                \node (0) at (135.04489, 16049.466) [above] {\tiny 149.04};
                \draw[\movLineColor] (149.04489, 15549.466) -- (135.04489, 16549.466);

                \node (1) at (167.05548, 30403.949) [above] {\tiny 167.05};
                \node (2) at (213.81911, 13479.453) [above] {\tiny 213.81};

                %\node (3) at (318.92282, 11948.519) [above] {\tiny 318.92};
                \node (3) at (300.92282, 11948.519) [above] {\tiny 318.92};
                \draw[\movLineColor] (318.92282, 11948.519) -- (300.92282, 12848.519);

                \node (4) at (341.01804, 28799.889) [above] {\tiny 341.01};

                %\node (5) at (359.02826, 24235.844) [above] {\tiny 359.02};
                \node (5) at (375.02826, 24635.844) [above] {\tiny 359.02};
                \draw[\movLineColor] (359.02826, 24235.844) -- (375.02826, 25235.844);

                %\node (6) at (404.45569, 5915.098) [above] {\tiny 404.45};
                \node (6) at (395.45569, 5915.098) [above] {\tiny 404.45};
                \draw[\movLineColor] (404.45569, 5915.098) -- (395.45569, 6615.098);

                \node (7) at (429.08765, 59781.848) [above] {\tiny 429.08};
                \node (8) at (486.75085, 6172.900) [above] {\tiny 486.75};

                % \node (9) at (564.11682, 5931.172) [above] {\tiny 564.11};
                \node (10) at (545.11682, 6631.172) [above] {\tiny 564.11};
                \draw[\movLineColor] (564.11682, 5931.172) -- (545.11682, 7531.172);

                \node (11) at (603.03772, 5612.658) [above] {\tiny 603.03};

                \draw [<->,thick] (0,60000) node (yaxis) [above] {\small \yAxisUnit}
                |- (650,0) node (xaxis) [right] {\small \xAxisUnit};
                \draw[thick] (149.04489, 0.0) -- (149.04489, 15549.466);
                \draw[thick] (167.05548, 0.0) -- (167.05548, 30403.949);
                \draw[thick] (173.28503, 0.0) -- (173.28503, 5523.645);
                \draw[thick] (193.11723, 0.0) -- (193.11723, 5522.963);
                \draw[thick] (203.12581, 0.0) -- (203.12581, 5886.653);
                %\draw[thick] (208.95285, 0.0) -- (208.95285, 47414.977);
                \draw[thick] (213.81911, 0.0) -- (213.81911, 13479.453);
                \draw[thick] (225.04166, 0.0) -- (225.04166, 8771.008);
                \draw[thick] (231.63631, 0.0) -- (231.63631, 5783.753);
                \draw[thick] (238.23830, 0.0) -- (238.23830, 5310.973);
                \draw[thick] (318.92282, 0.0) -- (318.92282, 11948.519);
                % Originial
                \draw[thick] (341.01804, 0.0) -- (341.01804, 28799.889);
                % Angepasst fuer Praesentationsfolien
                %\draw[thick] (341.01804, 0.0) -- (341.01804, 15799.889);
                \draw[thick] (358.97638, 0.0) -- (358.97638, 10126.957);
                % Original
                %\draw[thick] (359.02826, 0.0) -- (359.02826, 286235.844);
                \draw[thick] (359.02826, 0.0) -- (359.02826, 24235.844);
                \draw[thick] (429.08765, 0.0) -- (429.08765, 59781.848);
                % Originial
                % \draw[thick] (604.45569, 0.0) -- (604.45569, 5915.098);
                % \draw[thick] (686.75085, 0.0) -- (686.75085, 6172.900);
                % \draw[thick] (1064.11682, 0.0) -- (1064.11682, 5931.172);
                % \draw[thick] (1223.03772, 0.0) -- (1223.03772, 5612.658);
                % Angepasst fuer Praesentationsfolien
                \draw[thick] (404.45569, 0.0) -- (404.45569, 5915.098);
                \draw[thick] (486.75085, 0.0) -- (486.75085, 6172.900);
                \draw[thick] (564.11682, 0.0) -- (564.11682, 5931.172);
                \draw[thick] (603.03772, 0.0) -- (603.03772, 5612.658);
            \end{tikzpicture}
        }%
    \end{frame}
    %%%%% %%%%% %%%%% %%%%% %%%%% BEGIN Aminosäure Sequenzierung %%%%% %%%%% %%%%% %%%%% %%%%%



    %%%%% %%%%% %%%%% %%%%% %%%%% BEGIN De-Novo-Sequenzierung %%%%% %%%%% %%%%% %%%%% %%%%%
    \section{De-Novo-Sequenzierung}
    \newcommand{\AASequenzierungName}{Suche in Sequenzdatenbank}
    \begin{frame}{Sequenzierung und De-Novo-Sequenzierung}
        \onslide<1->{
            \centering
            \begin{tabulary}{0.95\textwidth}{c|c}
                \AASequenzierungName & De-Novo-Sequenzierung \\
                \midrule
                Datenbanken als Hilfsmittel & Ohne weitere Hilfsmittel\\
                & \\
                \makecell{Identifizierung von \textcolor{\highlightColor}{\emph{bekannten}}\\Sequenzen} & \makecell{Bestimmung \textcolor{\highlightColor}{\emph{unbekannter}}\\Sequenzen}\\
            \end{tabulary}
            \vspace*{0.5cm}
            \begin{itemize}
                \item<2-> De novo: lat. \gerquot{Von neuem}
            \end{itemize}
        }%
    \end{frame}


    \begin{frame}{De-Novo-Sequenzierung \dashAndSpace Übersicht}
        \begin{itemize}
            \item<1-> Zusätzliche Informationen notwendig
            \item<2-> Verwendung einer \textcolor{\highlightColor}{2. MS}
            \item<2-> Verfahren: Tandem-Massenspektrometrie MS2
        \end{itemize}
    \end{frame}


    \begin{frame}{De-Novo-Sequenzierung \dashAndSpace MS2}
        \newcommand{\rectangleFirstColor}{orange!80!}
        \newcommand{\rectangleSecondColor}{blue!90!}
        \newcommand{\rectangleThirdColor}{green!70!}
        \only<1-3>{
            \centering
            \begin{tikzpicture}[scale=1.1]
                \node (0) at (0.5, 0.5) {};
                \node (1) at (0.5, 1) {};
                \node (2) at (6.75, 0.5) {};
                \node (3) at (6.75, 1) {};
                \node (4) at (1.75, -0.5) {};
                \node (5) at (2.25, -0.5) {};
                \node (6) at (1.75, -1) {};
                \node (7) at (2.25, -1) {};
                \node (8) at (3.75, -1.5) {};
                \node (9) at (4.25, -1.5) {};
                \node (10) at (3.75, -2) {};
                \node (11) at (4.25, -2) {};
                \node (12) at (5.75, -2.5) {};
                \node (13) at (6.25, -2.5) {};
                \node (14) at (5.75, -3) {};
                \node (15) at (6.25, -3) {};
                \node (16) at (-0.25, 0.5) {MS1};
                \node (17) at (-0.25, -0.75) {MS2};
                \node (18) at (-0.25, -1.75) {MS2};
                \node (19) at (-0.25, -2.75) {MS2};
                \node (20) at (-0.75, -0.25) {};
                \node (21) at (-0.75, -1.25) {};
                \node (22) at (-0.75, -2.25) {};
                \node (23) at (-0.75, -3.25) {};
                \node (24) at (7, -2.25) {};
                \node (25) at (7, -3.25) {};
                \node (26) at (-0.75, 1.25) {};
                \node (27) at (7, -1.25) {};
                \node (28) at (7, -0.25) {};
                \node (29) at (7, 1.25) {};
                \node (30) at (0.25, 1.25) {};
                \node (31) at (0.25, -3.25) {};
                \node (32) at (0.5, 0) {};
                \node (33) at (6.75, 0) {};
                \node (34) at (3.5, 0.25) {};
                \node (35) at (3.5, 0.25) {\footnotesize $m/z$};
                \node (36) at (1.5, 1) {};
                \node (37) at (2.5, 1) {};
                \node (38) at (2.5, 0.5) {};
                \node (39) at (1.5, 0.5) {};
                \node (40) at (3.5, 1) {};
                \node (41) at (4.5, 1) {};
                \node (42) at (4.5, 0.5) {};
                \node (43) at (3.5, 0.5) {};
                \node (44) at (6.5, 1) {};
                \node (45) at (6.5, 0.5) {};
                \node (46) at (5.5, 0.5) {};
                \node (47) at (5.5, 1) {};

                % 1. Bereich im MS1 Rechteck
                \shade [left color=white,right color=\rectangleFirstColor] (36.center) rectangle (2, 0.5);
                \shade [left color=\rectangleFirstColor,right color=white] (2, 1) rectangle (38.center);

                % 2. Bereich im MS1 Rechteck
                \shade [left color=white,right color=\rectangleSecondColor] (40.center) rectangle (4, 0.5);
                \shade [left color=\rectangleSecondColor,right color=white] (4, 1) rectangle (42.center);

                % 3. Bereich im MS1 Rechteck
                \shade [left color=white,right color=\rectangleThirdColor] (47.center) rectangle (6, 0.5);
                \shade [left color=\rectangleThirdColor,right color=white] (6, 1) rectangle (45.center);

                \only<2,3>{
                    % 1. MS2 Rechteck
                    \shade [left color=white,right color=\rectangleFirstColor] (4.center) rectangle (2, -1);
                    \shade [left color=\rectangleFirstColor,right color=white] (2, -0.5) rectangle (7.center);

                    % 2. MS2 Rechteck
                    \shade [left color=white,right color=\rectangleSecondColor] (8.center) rectangle (4, -2);
                    \shade [left color=\rectangleSecondColor,right color=white] (4, -1.5) rectangle (11.center);

                    % 3. MS2 Rechteck
                    \shade [left color=white,right color=\rectangleThirdColor] (12.center) rectangle (6, -3);
                    \shade [left color=\rectangleThirdColor,right color=white] (6, -2.5) rectangle (15.center);
                }%

                \draw (1.center) to (3.center);
                \draw (0.center) to (2.center);
                \draw (3.center) to (2.center);
                \draw (1.center) to (0.center);
                \only<2,3>{
                    \draw (4.center) to (5.center);
                    \draw (4.center) to (6.center);
                    \draw (6.center) to (7.center);
                    \draw (5.center) to (7.center);
                    \draw (8.center) to (9.center);
                    \draw (8.center) to (10.center);
                    \draw (10.center) to (11.center);
                    \draw (9.center) to (11.center);
                    \draw (12.center) to (13.center);
                    \draw (12.center) to (14.center);
                    \draw (14.center) to (15.center);
                    \draw (13.center) to (15.center);
                }%
                \draw (22.center) to (23.center);
                \draw (23.center) to (25.center);
                \draw (22.center) to (24.center);
                \draw (24.center) to (25.center);
                \draw (22.center) to (21.center);
                \draw (21.center) to (20.center);
                \draw (20.center) to (26.center);
                \draw (26.center) to (29.center);
                \draw (29.center) to (28.center);
                \draw (28.center) to (27.center);
                \draw (27.center) to (24.center);
                \draw (27.center) to (21.center);
                \draw (20.center) to (28.center);
                \draw (30.center) to (31.center);
                \draw[<->, thick] (32.center) to (33.center);
            \end{tikzpicture}
        }%
        \begin{itemize}
            \item<3-> MS1 quasi eine \textcolor{\highlightColor}{Filterung}
            \item<3-> MS2 wird gezielt auf $m/z$ \textcolor{\highlightColor}{Intervalle} angewendet
        \end{itemize}
    \end{frame}
    %%%%% %%%%% %%%%% %%%%% %%%%% BEGIN De-Novo-Sequenzierung %%%%% %%%%% %%%%% %%%%% %%%%%



    %%%%% %%%%% %%%%% %%%%% %%%%% BEGIN pNovo+ Algorithmus %%%%% %%%%% %%%%% %%%%% %%%%%
    \section{pNovo+ Algorithmus}
    \begin{frame}{pNovo+ Algorithmus \dashAndSpace Übersicht}
        \begin{itemize}
            \item<1-> Erweiterung von pNovo
            \item<1-> Algorithmus für die De-Novo-Sequenzierung
            \item<1-> Auswertung von MS2-Spektren
            \item<1-> Rekonstruktion der AA-Sequenz
            \item<1-> Hilfsmittel: \textcolor{\highlightColor}{Spektrum-Graph}
        \end{itemize}
    \end{frame}

    \newcommand{\minWidth}{5.6cm}
    \newcommand{\minHeight}{1.0cm}
    \newcommand{\minWidthOverview}{2.3cm}
    \newcommand{\minHeightOverview}{0.325cm}
    \newcommand{\xStartOverview}{10}
    \newcommand{\yStartOverview}{15}
    \newcommand{\currElementOpacity}{1}
    \newcommand{\otherElementOpacity}{0.35}
    \newcommand{\stepSize}{1.6}
    \newcommand{\stepSizeOverview}{0.7}

    \newcommand{\colorStepOne}{orange!80!}
    \newcommand{\colorStepTwo}{yellow!60!}
    \newcommand{\colorStepThree}{green!60!}
    \newcommand{\colorStepFour}{cyan!60!}
    \newcommand{\colorStepFive}{magenta!60!}
    %%%%% %%%%% %%%%% BEGIN Uebersicht Grafik %%%%% %%%%% %%%%%
    \begin{frame}{pNovo+ Algorithmus \dashAndSpace Datenaufbereitung}
        \onslide<1-5>{
            \centering
            \begin{tikzpicture}
                \only<1>{
                    \node[rectangle,draw,text=black, fill=\colorStepOne, minimum width=\minWidth, minimum height=\minHeight, align=center, opacity=1] (r1) at (0,\stepSize * 0 * -1) {Hintergrundrauschen\\ verringern};
                    \node[rectangle,draw,text=black, fill=\colorStepTwo, minimum width=\minWidth, minimum height=\minHeight, align=center, opacity=0] (r2) at (0,\stepSize * 1 * -1) {Entfernung falscher Peaks};
                    \node[rectangle,draw,text=black, fill=\colorStepThree, minimum width=\minWidth, minimum height=\minHeight, align=center, opacity=0] (r3) at (0,\stepSize * 2 * -1) {Entfernung irrelevanter Peaks};
                    \node[rectangle,draw,text=black, fill=\colorStepFour, minimum width=\minWidth, minimum height=\minHeight, align=center, opacity=0] (r4) at (0,\stepSize * 3 * -1) {Normierung auf +1 Ladung};
                    \node[rectangle,draw,text=black, fill=\colorStepFive, minimum width=\minWidth, minimum height=\minHeight, align=center, opacity=0] (r5) at (0,\stepSize * 4 * -1) {Zusammenfassen};
                }%
                \only<2>{
                    \node[rectangle,draw,text=black, fill=\colorStepOne, minimum width=\minWidth, minimum height=\minHeight, align=center, opacity=1] (r1) at (0,\stepSize * 0 * -1) {Hintergrundrauschen\\ verringern};
                    \node[rectangle,draw,text=black, fill=\colorStepTwo, minimum width=\minWidth, minimum height=\minHeight, align=center, opacity=1] (r2) at (0,\stepSize * 1 * -1) {Entfernung falscher Peaks};
                    \node[rectangle,draw,text=black, fill=\colorStepThree, minimum width=\minWidth, minimum height=\minHeight, align=center, opacity=0] (r3) at (0,\stepSize * 2 * -1) {Entfernung irrelevanter Peaks};
                    \node[rectangle,draw,text=black, fill=\colorStepFour, minimum width=\minWidth, minimum height=\minHeight, align=center, opacity=0] (r4) at (0,\stepSize * 3 * -1) {Normierung auf +1 Ladung};
                    \node[rectangle,draw,text=black, fill=\colorStepFive, minimum width=\minWidth, minimum height=\minHeight, align=center, opacity=0] (r5) at (0,\stepSize * 4 * -1) {Zusammenfassen};
                    \draw[->, ultra thick, opacity=1] (r1) -- (r2);
                }%
                \only<3>{
                    \node[rectangle,draw,text=black, fill=\colorStepOne, minimum width=\minWidth, minimum height=\minHeight, align=center, opacity=1] (r1) at (0,\stepSize * 0 * -1) {Hintergrundrauschen\\ verringern};
                    \node[rectangle,draw,text=black, fill=\colorStepTwo, minimum width=\minWidth, minimum height=\minHeight, align=center, opacity=1] (r2) at (0,\stepSize * 1 * -1) {Entfernung falscher Peaks};
                    \node[rectangle,draw,text=black, fill=\colorStepThree, minimum width=\minWidth, minimum height=\minHeight, align=center, opacity=1] (r3) at (0,\stepSize * 2 * -1) {Entfernung irrelevanter Peaks};
                    \node[rectangle,draw,text=black, fill=\colorStepFour, minimum width=\minWidth, minimum height=\minHeight, align=center, opacity=0] (r4) at (0,\stepSize * 3 * -1) {Normierung auf +1 Ladung};
                    \node[rectangle,draw,text=black, fill=\colorStepFive, minimum width=\minWidth, minimum height=\minHeight, align=center, opacity=0] (r5) at (0,\stepSize * 4 * -1) {Zusammenfassen};
                    \draw[->, ultra thick, opacity=1] (r1) -- (r2);
                    \draw[->, ultra thick, opacity=1] (r2) -- (r3);
                }%
                \only<4>{
                    \node[rectangle,draw,text=black, fill=\colorStepOne, minimum width=\minWidth, minimum height=\minHeight, align=center, opacity=1] (r1) at (0,\stepSize * 0 * -1) {Hintergrundrauschen\\ verringern};
                    \node[rectangle,draw,text=black, fill=\colorStepTwo, minimum width=\minWidth, minimum height=\minHeight, align=center, opacity=1] (r2) at (0,\stepSize * 1 * -1) {Entfernung falscher Peaks};
                    \node[rectangle,draw,text=black, fill=\colorStepThree, minimum width=\minWidth, minimum height=\minHeight, align=center, opacity=1] (r3) at (0,\stepSize * 2 * -1) {Entfernung irrelevanter Peaks};
                    \node[rectangle,draw,text=black, fill=\colorStepFour, minimum width=\minWidth, minimum height=\minHeight, align=center, opacity=1] (r4) at (0,\stepSize * 3 * -1) {Normierung auf +1 Ladung};
                    \node[rectangle,draw,text=black, fill=\colorStepFive, minimum width=\minWidth, minimum height=\minHeight, align=center, opacity=0] (r5) at (0,\stepSize * 4 * -1) {Zusammenfassen};
                    \draw[->, ultra thick, opacity=1] (r1) -- (r2);
                    \draw[->, ultra thick, opacity=1] (r2) -- (r3);
                    \draw[->, ultra thick, opacity=1] (r3) -- (r4);
                }%
                \only<5>{
                    \node[rectangle,draw,text=black, fill=\colorStepOne, minimum width=\minWidth, minimum height=\minHeight, align=center, opacity=1] (r1) at (0,\stepSize * 0 * -1) {Hintergrundrauschen\\ verringern};
                    \node[rectangle,draw,text=black, fill=\colorStepTwo, minimum width=\minWidth, minimum height=\minHeight, align=center, opacity=1] (r2) at (0,\stepSize * 1 * -1) {Entfernung falscher Peaks};
                    \node[rectangle,draw,text=black, fill=\colorStepThree, minimum width=\minWidth, minimum height=\minHeight, align=center, opacity=1] (r3) at (0,\stepSize * 2 * -1) {Entfernung irrelevanter Peaks};
                    \node[rectangle,draw,text=black, fill=\colorStepFour, minimum width=\minWidth, minimum height=\minHeight, align=center, opacity=1] (r4) at (0,\stepSize * 3 * -1) {Normierung auf +1 Ladung};
                    \node[rectangle,draw,text=black, fill=\colorStepFive, minimum width=\minWidth, minimum height=\minHeight, align=center, opacity=1] (r5) at (0,\stepSize * 4 * -1) {Zusammenfassen};
                    \draw[->, ultra thick, opacity=1] (r1) -- (r2);
                    \draw[->, ultra thick, opacity=1] (r2) -- (r3);
                    \draw[->, ultra thick, opacity=1] (r3) -- (r4);
                    \draw[->, ultra thick, opacity=1] (r4) -- (r5);
                }%
            \end{tikzpicture}
        }%
    \end{frame}
    %%%%% %%%%% %%%%% END Uebersicht Grafik %%%%% %%%%% %%%%%

    %%%%% %%%%% %%%%% BEGIN ln() Grafik %%%%% %%%%% %%%%%
    \begin{frame}{pNovo+ Algorithmus \dashAndSpace Datenaufbereitung}
        \onslide<1->{
            \begin{minipage}[c]{0.775\textwidth}
                \centering
                \begin{itemize}
                    \item<1-> \textcolor{\highlightColor}{Überpriorisierung} fehlerhafter Daten vermeiden
                    \item<1-> Tool: $ln()$
                \end{itemize}
            \end{minipage}
        }%
        \onslide<1->{
            \begin{minipage}[c]{0.2\textwidth}
                \centering
                \begin{tikzpicture}[font=\tiny]
                    %\node[rectangle] at (0,0) {};
                    \node[rectangle,draw,text=black, fill=\colorStepOne,opacity=\currElementOpacity, minimum width=\minWidthOverview, minimum height=\minHeightOverview, align=center] (r1) at (\xStartOverview,\yStartOverview - \stepSizeOverview * 0) {Hintergrundrauschen};
                    \node[rectangle,draw,text=black, fill=\colorStepTwo,opacity=\otherElementOpacity, minimum width=\minWidthOverview, minimum height=\minHeightOverview, align=center] (r2) at (\xStartOverview,\yStartOverview - \stepSizeOverview * 1) {Falsche Peaks};
                    \node[rectangle,draw,text=black, fill=\colorStepThree, opacity=\otherElementOpacity,minimum width=\minWidthOverview, minimum height=\minHeightOverview, align=center] (r3) at (\xStartOverview,\yStartOverview - \stepSizeOverview * 2) {Irrelevante Peaks};
                    \node[rectangle,draw,text=black, fill=\colorStepFour, opacity=\otherElementOpacity,minimum width=\minWidthOverview, minimum height=\minHeightOverview, align=center] (r4) at (\xStartOverview,\yStartOverview - \stepSizeOverview * 3) {$+1$ Normierung};
                    \node[rectangle,draw,text=black, fill=\colorStepFive, opacity=\otherElementOpacity,minimum width=\minWidthOverview, minimum height=\minHeightOverview, align=center] (r5) at (\xStartOverview,\yStartOverview - \stepSizeOverview * 4) {Zusammenfassen};
                    \draw[->, ultra thick] (r1) -- (r2);
                    \draw[->, ultra thick] (r2) -- (r3);
                    \draw[->, ultra thick] (r3) -- (r4);
                    \draw[->, ultra thick] (r4) -- (r5);
                \end{tikzpicture}
            \end{minipage}
        }%
        \onslide<2-> {
            \begin{minipage}[l]{0.45\textwidth}
                \centering
                \begin{tikzpicture}[scale=\tikzScale, baseline=(current bounding box.center)]
                    \draw[thick] (0.2, 0.0) -- (0.2, 2.3);
                    \draw[thick] (0.382, 0.0) -- (0.382, 1.7);
                    \draw[thick] (0.476, 0.0) -- (0.476, 2.7);
                    \draw[thick] (0.456, 0.0) -- (0.456, 1.8);
                    \draw[thick] (0.6859999999999999, 0.0) -- (0.6859999999999999, 2.7);
                    \draw[thick] (0.6839999999999999, 0.0) -- (0.6839999999999999, 1.8);
                    \draw[thick] (0.752, 0.0) -- (0.752, 1.1);
                    \draw[thick] (0.8200000000000001, 0.0) -- (0.8200000000000001, 2.2);
                    \draw[thick] (1.076, 0.0) -- (1.076, 1.5);
                    \draw[thick] (1.16, 0.0) -- (1.16, 1.9);
                    \draw[thick] (1.2120000000000002, 0.0) -- (1.2120000000000002, 2.0);
                    \draw[thick] (1.28, 0.0) -- (1.28, 1.9);
                    \draw[thick] (1.452, 0.0) -- (1.452, 1.3);
                    \draw[thick] (1.426, 0.0) -- (1.426, 1.9);
                    \draw[thick] (1.548, 0.0) -- (1.548, 1.9);
                    \draw[thick] (1.6740000000000002, 0.0) -- (1.6740000000000002, 1.5);
                    \draw[thick] (1.788, 0.0) -- (1.788, 2.5);
                    \draw[thick] (1.856, 0.0) -- (1.856, 2.3);
                    \draw[thick] (2.036, 0.0) -- (2.036, 1.7);
                    \draw[thick] (2.142, 0.0) -- (2.142, 1.6);
                    \draw[thick] (2.2520000000000002, 0.0) -- (2.2520000000000002, 2.0);
                    \draw[thick] (2.386, 0.0) -- (2.386, 1.6);
                    \draw[thick] (2.488, 0.0) -- (2.488, 2.9);
                    \draw[thick] (2.4739999999999998, 0.0) -- (2.4739999999999998, 2.7);
                    \draw[thick] (2.504, 0.0) -- (2.504, 2.0);
                    \draw[thick] (2.682, 0.0) -- (2.682, 2.0);
                    \draw[thick] (2.702, 0.0) -- (2.702, 2.5);
                    \draw[thick] (2.9259999999999997, 0.0) -- (2.9259999999999997, 2.8);
                    \draw[thick] (3.024, 0.0) -- (3.024, 2.4);
                    \draw[thick] (3.096, 0.0) -- (3.096, 1.8);
                    \draw[thick] (3.244, 0.0) -- (3.244, 2.6);
                    \draw[thick] (3.362, 0.0) -- (3.362, 1.9);
                    \draw[thick] (3.46, 0.0) -- (3.46, 2.3);
                    \draw[thick] (3.516, 0.0) -- (3.516, 1.1);
                    \draw[thick] (3.584, 0.0) -- (3.584, 1.8);
                    \draw[thick] (3.652, 0.0) -- (3.652, 2.0);
                    \draw[thick] (3.838, 0.0) -- (3.838, 1.5);
                    \draw[thick] (3.8819999999999997, 0.0) -- (3.8819999999999997, 2.6);
                    \draw[thick] (4.088, 0.0) -- (4.088, 2.6);
                    \draw[thick] (4.046, 0.0) -- (4.046, 1.1);
                    \draw[thick] (4.167999999999999, 0.0) -- (4.167999999999999, 2.0);
                    \draw[thick] (4.266, 0.0) -- (4.266, 2.4);
                    \draw[thick] (4.38, 0.0) -- (4.38, 1.1);
                    \draw[thick] (4.456, 0.0) -- (4.456, 2.2);
                    \draw[thick] (4.644, 0.0) -- (4.644, 2.6);
                    \draw[thick] (4.675999999999999, 0.0) -- (4.675999999999999, 2.5);
                    \draw[thick] (4.898000000000001, 0.0) -- (4.898000000000001, 1.2);

                    \draw [<->,thick] (0,\yAxisHeight) node (yaxis) [above] {\yAxisUnit} |- (\xAxisLength,0) node (xaxis) [right] {\xAxisUnit};
                \end{tikzpicture}
            \end{minipage}
        }%
        \onslide<3->{
            \textbf{$\rightarrow$}
        }%
        \onslide<3->{
            \begin{minipage}[r]{0.45\textwidth}
                \centering
                \begin{tikzpicture}[scale=\tikzScale, baseline=(current bounding box.center)]
                    \draw[thick] (0.2, 0.0) -- (0.2, {ln(2.3)});
                    \draw[thick] (0.382, 0.0) -- (0.382, {ln(1.7)});
                    \draw[thick] (0.476, 0.0) -- (0.476, {ln(2.7)});
                    \draw[thick] (0.456, 0.0) -- (0.456, {ln(1.8)});
                    \draw[thick] (0.6859999999999999, 0.0) -- (0.6859999999999999, {ln(2.7)});
                    \draw[thick] (0.6839999999999999, 0.0) -- (0.6839999999999999, {ln(1.8)});
                    \draw[thick] (0.752, 0.0) -- (0.752, {ln(1.1)});
                    \draw[thick] (0.8200000000000001, 0.0) -- (0.8200000000000001, {ln(2.2)});
                    \draw[thick] (1.076, 0.0) -- (1.076, {ln(1.5)});
                    \draw[thick] (1.16, 0.0) -- (1.16, {ln(1.9)});
                    \draw[thick] (1.2120000000000002, 0.0) -- (1.2120000000000002, {ln(2.0)});
                    \draw[thick] (1.28, 0.0) -- (1.28, {ln(1.9)});
                    \draw[thick] (1.452, 0.0) -- (1.452, {ln(1.3)});
                    \draw[thick] (1.426, 0.0) -- (1.426, {ln(1.9)});
                    \draw[thick] (1.548, 0.0) -- (1.548, {ln(1.9)});
                    \draw[thick] (1.6740000000000002, 0.0) -- (1.6740000000000002, {ln(1.5)});
                    \draw[thick] (1.788, 0.0) -- (1.788, {ln(2.5)});
                    \draw[thick] (1.856, 0.0) -- (1.856, {ln(2.3)});
                    \draw[thick] (2.036, 0.0) -- (2.036, {ln(1.7)});
                    \draw[thick] (2.142, 0.0) -- (2.142, {ln(1.6)});
                    \draw[thick] (2.2520000000000002, 0.0) -- (2.2520000000000002, {ln(2.0)});
                    \draw[thick] (2.386, 0.0) -- (2.386, {ln(1.6)});
                    \draw[thick] (2.488, 0.0) -- (2.488, {ln(2.9)});
                    \draw[thick] (2.4739999999999998, 0.0) -- (2.4739999999999998, {ln(2.7)});
                    \draw[thick] (2.504, 0.0) -- (2.504, {ln(2.0)});
                    \draw[thick] (2.682, 0.0) -- (2.682, {ln(2.0)});
                    \draw[thick] (2.702, 0.0) -- (2.702, {ln(2.5)});
                    \draw[thick] (2.9259999999999997, 0.0) -- (2.9259999999999997, {ln(2.8)});
                    \draw[thick] (3.024, 0.0) -- (3.024, {ln(2.4)});
                    \draw[thick] (3.096, 0.0) -- (3.096, {ln(1.8)});
                    \draw[thick] (3.244, 0.0) -- (3.244, {ln(2.6)});
                    \draw[thick] (3.362, 0.0) -- (3.362, {ln(1.9)});
                    \draw[thick] (3.46, 0.0) -- (3.46, {ln(2.3)});
                    \draw[thick] (3.516, 0.0) -- (3.516, {ln(1.1)});
                    \draw[thick] (3.584, 0.0) -- (3.584, {ln(1.8)});
                    \draw[thick] (3.652, 0.0) -- (3.652, {ln(2.0)});
                    \draw[thick] (3.838, 0.0) -- (3.838, {ln(1.5)});
                    \draw[thick] (3.8819999999999997, 0.0) -- (3.8819999999999997, {ln(2.6)});
                    \draw[thick] (4.088, 0.0) -- (4.088, {ln(2.6)});
                    \draw[thick] (4.046, 0.0) -- (4.046, {ln(1.1)});
                    \draw[thick] (4.167999999999999, 0.0) -- (4.167999999999999, {ln(2.0)});
                    \draw[thick] (4.266, 0.0) -- (4.266, {ln(2.4)});
                    \draw[thick] (4.38, 0.0) -- (4.38, {ln(1.1)});
                    \draw[thick] (4.456, 0.0) -- (4.456, {ln(2.2)});
                    \draw[thick] (4.644, 0.0) -- (4.644, {ln(2.6)});
                    \draw[thick] (4.675999999999999, 0.0) -- (4.675999999999999, {ln(2.5)});
                    \draw[thick] (4.898000000000001, 0.0) -- (4.898000000000001, {ln(1.2)});

                    \draw [<->,thick] (0,\yAxisHeight) node (yaxis) [above] {\yAxisUnit} |- (\xAxisLength,0) node (xaxis) [right] {\xAxisUnit};
                \end{tikzpicture}
            \end{minipage}
        }%
    \end{frame}
    %%%%% %%%%% %%%%% END ln() Grafik %%%%% %%%%% %%%%%

    %%%%% %%%%% %%%%% BEGIN Monoisotopische Masse Grafik %%%%% %%%%% %%%%%
    \newcommand{\colorMonoisotopicMassPeak}{magenta}
    \newcommand{\colorPeakWithChargeDistance}{blue!85!}
    \newcommand{\colorSpyText}{black}
    \begin{frame}{pNovo+ Algorithmus \dashAndSpace Datenaufbereitung}
        \onslide<1->{
            \begin{minipage}[c]{0.775\textwidth}
                \centering
                % Farbentechnisch erkennbar machen, was Monoisotopische Peaks und was "Nachbar-Peaks" sind
                \begin{itemize}
                    \item<1-> \textcolor{\colorMonoisotopicMassPeak}{Monoisotopische Peaks} auswählen
                    % \item<2-> $\Rightarrow$ Peaks, die garantiert von AA stammen % (monoisotopische Masse)
                    \item<2-> Peaks mit definierten \textcolor{\colorPeakWithChargeDistance}{Abstand} auswählen
                    \item<5-> \gerquot{Charge state} Zuweisung:
                    \begin{itemize}
                        \item<5-> \textcolor{\colorMonoisotopicMassPeak}{Monoisotopischer Peak}: $+1$
                        \item<5-> \textcolor{\colorPeakWithChargeDistance}{Abstand}: steigend mit $+1$ pro Schritt
                    \end{itemize}
                \end{itemize}
            \end{minipage}
        }%
        \onslide<1->{
            \begin{minipage}[c]{0.2\textwidth}
                \centering
                \begin{tikzpicture}[font=\tiny]
                    %\node[rectangle] at (0,0) {};
                    \node[rectangle,draw,text=black, fill=\colorStepOne,opacity=\otherElementOpacity, minimum width=\minWidthOverview, minimum height=\minHeightOverview, align=center] (r1) at (\xStartOverview,\yStartOverview - \stepSizeOverview * 0) {Hintergrundrauschen};
                    \node[rectangle,draw,text=black, fill=\colorStepTwo,opacity=\currElementOpacity, minimum width=\minWidthOverview, minimum height=\minHeightOverview, align=center] (r2) at (\xStartOverview,\yStartOverview - \stepSizeOverview * 1) {Falsche Peaks};
                    \node[rectangle,draw,text=black, fill=\colorStepThree, opacity=\otherElementOpacity,minimum width=\minWidthOverview, minimum height=\minHeightOverview, align=center] (r3) at (\xStartOverview,\yStartOverview - \stepSizeOverview * 2) {Irrelevante Peaks};
                    \node[rectangle,draw,text=black, fill=\colorStepFour, opacity=\otherElementOpacity,minimum width=\minWidthOverview, minimum height=\minHeightOverview, align=center] (r4) at (\xStartOverview,\yStartOverview - \stepSizeOverview * 3) {$+1$ Normierung};
                    \node[rectangle,draw,text=black, fill=\colorStepFive, opacity=\otherElementOpacity,minimum width=\minWidthOverview, minimum height=\minHeightOverview, align=center] (r5) at (\xStartOverview,\yStartOverview - \stepSizeOverview * 4) {Zusammenfassen};
                    \draw[->, ultra thick] (r1) -- (r2);
                    \draw[->, ultra thick] (r2) -- (r3);
                    \draw[->, ultra thick] (r3) -- (r4);
                    \draw[->, ultra thick] (r4) -- (r5);
                \end{tikzpicture}
            \end{minipage}
        }%
        \onslide<3-> {
            \begin{minipage}[t]{.45\linewidth}
                \centering
                \begin{tikzpicture}[scale=\tikzScale, baseline=(current bounding box.center)]
                    \draw[thick] (0.2, 0.0) -- (0.2, {ln(2.3)});
                    \draw[color=\colorPeakWithChargeDistance,thick] (0.382, 0.0) -- (0.382, {ln(1.7)});
                    \draw[color=\colorPeakWithChargeDistance,thick] (0.476, 0.0) -- (0.476, {ln(2.7)});
                    \draw[color=\colorPeakWithChargeDistance,thick] (0.456, 0.0) -- (0.456, {ln(1.8)});
                    \draw[color=\colorPeakWithChargeDistance,thick] (0.6859999999999999, 0.0) -- (0.6859999999999999, {ln(2.7)});
                    \draw[color=\colorMonoisotopicMassPeak,thick] (0.6839999999999999, 0.0) -- (0.6839999999999999, {ln(1.8)});
                    \draw[thick] (0.752, 0.0) -- (0.752, {ln(1.1)});
                    \draw[thick] (0.8200000000000001, 0.0) -- (0.8200000000000001, {ln(2.2)});
                    \draw[thick] (1.076, 0.0) -- (1.076, {ln(1.5)});
                    \draw[thick] (1.16, 0.0) -- (1.16, {ln(1.9)});
                    \draw[thick] (1.2120000000000002, 0.0) -- (1.2120000000000002, {ln(2.0)});
                    \draw[thick] (1.28, 0.0) -- (1.28, {ln(1.9)});
                    \draw[color=\colorPeakWithChargeDistance,thick] (1.452, 0.0) -- (1.452, {ln(1.3)});
                    \draw[color=\colorPeakWithChargeDistance,thick] (1.426, 0.0) -- (1.426, {ln(1.9)});
                    \draw[color=\colorPeakWithChargeDistance] (1.548, 0.0) -- (1.548, {ln(1.9)});
                    \draw[color=\colorPeakWithChargeDistance,thick] (1.6740000000000002, 0.0) -- (1.6740000000000002, {ln(1.5)});
                    \draw[color=\colorMonoisotopicMassPeak,thick] (1.788, 0.0) -- (1.788, {ln(2.5)});
                    \draw[thick] (1.856, 0.0) -- (1.856, {ln(2.3)});
                    \draw[thick] (2.036, 0.0) -- (2.036, {ln(1.7)});
                    \draw[thick] (2.142, 0.0) -- (2.142, {ln(1.6)});
                    \draw[thick] (2.2520000000000002, 0.0) -- (2.2520000000000002, {ln(2.0)});
                    \draw[thick] (2.386, 0.0) -- (2.386, {ln(1.6)});
                    \draw[color=\colorPeakWithChargeDistance,thick] (2.488, 0.0) -- (2.488, {ln(2.9)});
                    \draw[thick] (2.4739999999999998, 0.0) -- (2.4739999999999998, {ln(2.7)});
                    \draw[color=\colorPeakWithChargeDistance,thick] (2.504, 0.0) -- (2.504, {ln(2.0)});
                    \draw[color=\colorPeakWithChargeDistance,thick] (2.682, 0.0) -- (2.682, {ln(2.0)});
                    \draw[color=\colorMonoisotopicMassPeak,thick] (2.702, 0.0) -- (2.702, {ln(2.5)});
                    \draw[thick] (2.9259999999999997, 0.0) -- (2.9259999999999997, {ln(2.8)});
                    \draw[thick] (3.024, 0.0) -- (3.024, {ln(2.4)});
                    \draw[thick] (3.096, 0.0) -- (3.096, {ln(1.8)});
                    \draw[thick] (3.244, 0.0) -- (3.244, {ln(2.6)});
                    \draw[thick] (3.362, 0.0) -- (3.362, {ln(1.9)});
                    \draw[color=\colorPeakWithChargeDistance,thick] (3.46, 0.0) -- (3.46, {ln(2.3)});
                    \draw[color=\colorPeakWithChargeDistance,thick] (3.516, 0.0) -- (3.516, {ln(1.1)});
                    \draw[color=\colorPeakWithChargeDistance,thick] (3.584, 0.0) -- (3.584, {ln(1.8)});
                    \draw[color=\colorPeakWithChargeDistance,thick] (3.652, 0.0) -- (3.652, {ln(2.0)});
                    \draw[color=\colorPeakWithChargeDistance,thick] (3.838, 0.0) -- (3.838, {ln(1.5)});
                    \draw[color=\colorMonoisotopicMassPeak,thick] (3.8819999999999997, 0.0) -- (3.8819999999999997, {ln(2.6)});
                    \draw[thick] (4.088, 0.0) -- (4.088, {ln(2.6)});
                    \draw[thick] (4.046, 0.0) -- (4.046, {ln(1.1)});
                    \draw[thick] (4.167999999999999, 0.0) -- (4.167999999999999, {ln(2.0)});
                    \draw[thick] (4.266, 0.0) -- (4.266, {ln(2.4)});
                    \draw[color=\colorPeakWithChargeDistance,thick] (4.38, 0.0) -- (4.38, {ln(1.1)});
                    \draw[color=\colorPeakWithChargeDistance,thick] (4.456, 0.0) -- (4.456, {ln(2.2)});
                    \draw[color=\colorPeakWithChargeDistance,thick] (4.644, 0.0) -- (4.644, {ln(2.6)});
                    \draw[color=\colorPeakWithChargeDistance,thick] (4.675999999999999, 0.0) -- (4.675999999999999, {ln(2.5)});
                    \draw[color=\colorMonoisotopicMassPeak,thick] (4.898000000000001, 0.0) -- (4.898000000000001, {ln(1.2)});

                    \draw [<->,thick] (0,\yAxisHeight) node (yaxis) [above] {\yAxisUnit} |- (\xAxisLength,0) node (xaxis) [right] {\xAxisUnit};
                \end{tikzpicture}%
            \end{minipage}%
            % Hier NICHT anwenden!
            %\hfill
        }%
        \onslide<4-> {
            \textbf{$\rightarrow$}
        }%
        \only<4,5> {
            \begin{minipage}[t]{.45\linewidth}
                \centering
                \begin{tikzpicture}[scale=\tikzScale, baseline=(current bounding box.center), spy using outlines={rectangle, magnification=2, size=1.5cm, connect spies}]
                    \draw[color=\colorPeakWithChargeDistance,thick] (0.382, 0.0) -- (0.382, {ln(1.7)});
                    \draw[color=\colorPeakWithChargeDistance,thick] (0.476, 0.0) -- (0.476, {ln(2.7)});
                    \draw[color=\colorPeakWithChargeDistance,thick] (0.456, 0.0) -- (0.456, {ln(1.8)});
                    \draw[color=\colorPeakWithChargeDistance,thick] (0.6859999999999999, 0.0) -- (0.6859999999999999, {ln(2.7)});
                    \draw[color=\colorMonoisotopicMassPeak,thick] (0.6839999999999999, 0.0) -- (0.6839999999999999, {ln(1.8)});

                    \only<4>{
                        \draw[color=\colorPeakWithChargeDistance,thick] (1.452, 0.0) -- (1.452, {ln(1.3)});
                        \draw[color=\colorPeakWithChargeDistance,thick] (1.426, 0.0) -- (1.426, {ln(1.9)});
                        \draw[color=\colorPeakWithChargeDistance,thick] (1.548, 0.0) -- (1.548, {ln(1.9)});
                        \draw[color=\colorPeakWithChargeDistance,thick] (1.6740000000000002, 0.0) -- (1.6740000000000002, {ln(1.5)});
                        \draw[color=\colorMonoisotopicMassPeak,thick] (1.788, 0.0) -- (1.788, {ln(2.5)});
                    }%

                    \only<5>{
                        \draw[color=\colorPeakWithChargeDistance,thick] (1.452, 0.0) -- (1.452, {ln(1.3)}); % NEIN hier fehlt KEIN Node!
                        \draw[color=\colorPeakWithChargeDistance,thick] (1.426, 0.0) -- (1.426, {ln(1.9)}) node [left, xshift=+0.1cm, yshift=-0.1cm] {\fontsize{5}{6}\selectfont \textcolor{\colorSpyText}{+4}};
                        \draw[color=\colorPeakWithChargeDistance,thick] (1.548, 0.0) -- (1.548, {ln(1.9)}) node [above, yshift=-0.1cm] {\fontsize{5}{6}\selectfont \textcolor{\colorSpyText}{+3}};
                        \draw[color=\colorPeakWithChargeDistance,thick] (1.6740000000000002, 0.0) -- (1.6740000000000002, {ln(1.5)}) node [above, yshift=-0.1cm] {\fontsize{5}{6}\selectfont \textcolor{\colorSpyText}{+2}};
                        \draw[color=\colorMonoisotopicMassPeak,thick] (1.788, 0.0) -- (1.788, {ln(2.5)}) node [above, yshift=-0.1cm] {\fontsize{5}{6}\selectfont \textcolor{\colorSpyText}{+1}};
                    }%

                    \draw[color=\colorPeakWithChargeDistance,thick] (2.488, 0.0) -- (2.488, {ln(2.9)});
                    \draw[color=\colorPeakWithChargeDistance,thick] (2.504, 0.0) -- (2.504, {ln(2.0)});
                    \draw[color=\colorPeakWithChargeDistance,thick] (2.682, 0.0) -- (2.682, {ln(2.0)});
                    \draw[color=\colorMonoisotopicMassPeak,thick] (2.702, 0.0) -- (2.702, {ln(2.5)});

                    \draw[color=\colorPeakWithChargeDistance,thick] (3.46, 0.0) -- (3.46, {ln(2.3)});
                    \draw[color=\colorPeakWithChargeDistance,thick] (3.516, 0.0) -- (3.516, {ln(1.1)});
                    \draw[color=\colorPeakWithChargeDistance,thick] (3.584, 0.0) -- (3.584, {ln(1.8)});
                    \draw[color=\colorPeakWithChargeDistance,thick] (3.652, 0.0) -- (3.652, {ln(2.0)});
                    \draw[color=\colorPeakWithChargeDistance,thick] (3.838, 0.0) -- (3.838, {ln(1.5)});
                    \draw[color=\colorMonoisotopicMassPeak,thick] (3.8819999999999997, 0.0) -- (3.8819999999999997, {ln(2.6)});

                    \draw[color=\colorPeakWithChargeDistance,thick] (4.38, 0.0) -- (4.38, {ln(1.1)});
                    \draw[color=\colorPeakWithChargeDistance,thick] (4.456, 0.0) -- (4.456, {ln(2.2)});
                    \draw[color=\colorPeakWithChargeDistance,thick] (4.644, 0.0) -- (4.644, {ln(2.6)});
                    \draw[color=\colorPeakWithChargeDistance,thick] (4.675999999999999, 0.0) -- (4.675999999999999, {ln(2.5)});
                    \draw[color=\colorMonoisotopicMassPeak,thick] (4.898000000000001, 0.0) -- (4.898000000000001, {ln(1.2)});

                    \only<5>{
                        \spy [red] on (1.175,0.6) in node [left] at (6,2.25);
                    }%

                    \draw [<->,thick] (0,\yAxisHeight) node (yaxis) [above] {\yAxisUnit} |- (\xAxisLength,0) node (xaxis) [right] {\xAxisUnit};
                \end{tikzpicture}
            \end{minipage}
            \hfill
        }%
    \end{frame}
    %%%%% %%%%% %%%%% END Monoisotopische Masse Grafik %%%%% %%%%% %%%%%

    %%%%% %%%%% %%%%% BEGIN Irrelevanter Bereich Grafik %%%%% %%%%% %%%%%
    \begin{frame}{pNovo+ Algorithmus \dashAndSpace Datenaufbereitung}
        \onslide<1->{
            \begin{minipage}[c]{0.775\textwidth}
                \centering
                \begin{itemize}
                    \item<1-> Peaks aus \textcolor{\highlightColor}{irrelevantem Intervall} entfernen
                    \item<1-> $m/z$ Bereiche, die garantiert unwichtig sind
                \end{itemize}
            \end{minipage}
        }%
        \onslide<1->{
            \begin{minipage}[c]{0.2\textwidth}
                \centering
                \begin{tikzpicture}[font=\tiny]
                    %\node[rectangle] at (0,0) {};
                    \node[rectangle,draw,text=black, fill=\colorStepOne,opacity=\otherElementOpacity, minimum width=\minWidthOverview, minimum height=\minHeightOverview, align=center] (r1) at (\xStartOverview,\yStartOverview - \stepSizeOverview * 0) {Hintergrundrauschen};
                    \node[rectangle,draw,text=black, fill=\colorStepTwo,opacity=\otherElementOpacity, minimum width=\minWidthOverview, minimum height=\minHeightOverview, align=center] (r2) at (\xStartOverview,\yStartOverview - \stepSizeOverview * 1) {Falsche Peaks};
                    \node[rectangle,draw,text=black, fill=\colorStepThree, opacity=\currElementOpacity,minimum width=\minWidthOverview, minimum height=\minHeightOverview, align=center] (r3) at (\xStartOverview,\yStartOverview - \stepSizeOverview * 2) {Irrelevante Peaks};
                    \node[rectangle,draw,text=black, fill=\colorStepFour, opacity=\otherElementOpacity,minimum width=\minWidthOverview, minimum height=\minHeightOverview, align=center] (r4) at (\xStartOverview,\yStartOverview - \stepSizeOverview * 3) {$+1$ Normierung};
                    \node[rectangle,draw,text=black, fill=\colorStepFive, opacity=\otherElementOpacity,minimum width=\minWidthOverview, minimum height=\minHeightOverview, align=center] (r5) at (\xStartOverview,\yStartOverview - \stepSizeOverview * 4) {Zusammenfassen};
                    \draw[->, ultra thick] (r1) -- (r2);
                    \draw[->, ultra thick] (r2) -- (r3);
                    \draw[->, ultra thick] (r3) -- (r4);
                    \draw[->, ultra thick] (r4) -- (r5);
                \end{tikzpicture}
            \end{minipage}
        }%
        \onslide<2-> {
            \begin{minipage}[t]{.45\linewidth}
                \centering
                \begin{tikzpicture}[scale=\tikzScale, baseline=(current bounding box.center)]
                    \fill[red!25!,opacity=.75] (0,0) rectangle (1,\yAxisHeight-\axisColorOffset);
                    \fill[red!25!,opacity=.75] (\xAxisLength-1,0) rectangle (\xAxisLength-\axisColorOffset,\yAxisHeight-\axisColorOffset);
                    \fill[green!25!,opacity=.75] (1,0) rectangle (\xAxisLength-1,\yAxisHeight-\axisColorOffset);

                    \draw[thick] (0.382, 0.0) -- (0.382, {ln(1.7)});
                    \draw[thick] (0.476, 0.0) -- (0.476, {ln(2.7)});
                    \draw[thick] (0.456, 0.0) -- (0.456, {ln(1.8)});
                    \draw[thick] (0.6859999999999999, 0.0) -- (0.6859999999999999, {ln(2.7)});
                    \draw[thick] (0.6839999999999999, 0.0) -- (0.6839999999999999, {ln(1.8)});
                    \draw[thick] (1.452, 0.0) -- (1.452, {ln(1.3)});
                    \draw[thick] (1.426, 0.0) -- (1.426, {ln(1.9)});
                    \draw[thick] (1.548, 0.0) -- (1.548, {ln(1.9)});
                    \draw[thick] (1.6740000000000002, 0.0) -- (1.6740000000000002, {ln(1.5)});
                    \draw[thick] (1.788, 0.0) -- (1.788, {ln(2.5)});
                    \draw[thick] (2.488, 0.0) -- (2.488, {ln(2.9)});
                    \draw[thick] (2.504, 0.0) -- (2.504, {ln(2.0)});
                    \draw[thick] (2.682, 0.0) -- (2.682, {ln(2.0)});
                    \draw[thick] (2.702, 0.0) -- (2.702, {ln(2.5)});
                    \draw[thick] (3.46, 0.0) -- (3.46, {ln(2.3)});
                    \draw[thick] (3.516, 0.0) -- (3.516, {ln(1.1)});
                    \draw[thick] (3.584, 0.0) -- (3.584, {ln(1.8)});
                    \draw[thick] (3.652, 0.0) -- (3.652, {ln(2.0)});
                    \draw[thick] (3.838, 0.0) -- (3.838, {ln(1.5)});
                    \draw[thick] (3.8819999999999997, 0.0) -- (3.8819999999999997, {ln(2.6)});
                    \draw[thick] (4.38, 0.0) -- (4.38, {ln(1.1)});
                    \draw[thick] (4.456, 0.0) -- (4.456, {ln(2.2)});
                    \draw[thick] (4.644, 0.0) -- (4.644, {ln(2.6)});
                    \draw[thick] (4.675999999999999, 0.0) -- (4.675999999999999, {ln(2.5)});
                    \draw[thick] (4.898000000000001, 0.0) -- (4.898000000000001, {ln(1.2)});

                    \draw [<->,thick] (0,\yAxisHeight) node (yaxis) [above] {\yAxisUnit} |- (\xAxisLength,0) node (xaxis) [right] {\xAxisUnit};
                \end{tikzpicture}%
            \end{minipage}%
        }%
        \onslide<3-> {
            \textbf{$\rightarrow$}
        }%
        \onslide<3->{
            \begin{minipage}[t]{.45\linewidth}
                \centering
                \begin{tikzpicture}[scale=\tikzScale, baseline=(current bounding box.center)]
                    \fill[red!25!,opacity=.75] (0,0) rectangle (1,\yAxisHeight-\axisColorOffset);
                    \fill[red!25!,opacity=.75] (\xAxisLength-1,0) rectangle (\xAxisLength-\axisColorOffset,\yAxisHeight-\axisColorOffset);
                    \fill[green!25!,opacity=.75] (1,0) rectangle (\xAxisLength-1,\yAxisHeight-\axisColorOffset);

                    %\draw[thick] (0.382, 0.0) -- (0.382, {ln(1.7)});
                    %\draw[thick] (0.476, 0.0) -- (0.476, {ln(2.7)});
                    %\draw[thick] (0.456, 0.0) -- (0.456, {ln(1.8)});
                    %\draw[thick] (0.6859999999999999, 0.0) -- (0.6859999999999999, {ln(2.7)});
                    %\draw[thick] (0.6839999999999999, 0.0) -- (0.6839999999999999, {ln(1.8)});
                    \draw[thick] (1.452, 0.0) -- (1.452, {ln(1.3)});
                    \draw[thick] (1.426, 0.0) -- (1.426, {ln(1.9)});
                    \draw[thick] (1.548, 0.0) -- (1.548, {ln(1.9)});
                    \draw[thick] (1.6740000000000002, 0.0) -- (1.6740000000000002, {ln(1.5)});
                    \draw[thick] (1.788, 0.0) -- (1.788, {ln(2.5)});
                    \draw[thick] (2.488, 0.0) -- (2.488, {ln(2.9)});
                    \draw[thick] (2.504, 0.0) -- (2.504, {ln(2.0)});
                    \draw[thick] (2.682, 0.0) -- (2.682, {ln(2.0)});
                    \draw[thick] (2.702, 0.0) -- (2.702, {ln(2.5)});
                    \draw[thick] (3.46, 0.0) -- (3.46, {ln(2.3)});
                    \draw[thick] (3.516, 0.0) -- (3.516, {ln(1.1)});
                    \draw[thick] (3.584, 0.0) -- (3.584, {ln(1.8)});
                    \draw[thick] (3.652, 0.0) -- (3.652, {ln(2.0)});
                    \draw[thick] (3.838, 0.0) -- (3.838, {ln(1.5)});
                    \draw[thick] (3.8819999999999997, 0.0) -- (3.8819999999999997, {ln(2.6)});
                    %\draw[thick] (4.38, 0.0) -- (4.38, {ln(1.1)});
                    %\draw[thick] (4.456, 0.0) -- (4.456, {ln(2.2)});
                    %\draw[thick] (4.644, 0.0) -- (4.644, {ln(2.6)});
                    %\draw[thick] (4.675999999999999, 0.0) -- (4.675999999999999, {ln(2.5)});
                    %\draw[thick] (4.898000000000001, 0.0) -- (4.898000000000001, {ln(1.2)});

                    \draw [<->,thick] (0,\yAxisHeight) node (yaxis) [above] {\yAxisUnit} |- (\xAxisLength,0) node (xaxis) [right] {\xAxisUnit};
                \end{tikzpicture}%
            \end{minipage}%
        }%
    \end{frame}
    %%%%% %%%%% %%%%% END Irrelevanter Bereich Grafik %%%%% %%%%% %%%%%

    %%%%% %%%%% %%%%% BEGIN +1 Normierung Grafik %%%%% %%%%% %%%%%
    \newcommand{\normierungOffset}{0.35}
    \begin{frame}{pNovo+ Algorithmus \dashAndSpace Datenaufbereitung}
        \onslide<1->{
            \begin{minipage}[c]{0.775\textwidth}
                \centering
                \begin{itemize}
                    \item<1-> Peaks auf Charge state $+1$ \textcolor{\highlightColor}{normieren}
                    \item<1-> $\Rightarrow$ Verschiebung nach rechts auf $m/z$ Achse
                \end{itemize}
            \end{minipage}
        }%
        \onslide<1->{
            \begin{minipage}[c]{0.2\textwidth}
                \centering
                \begin{tikzpicture}[font=\tiny]
                    %\node[rectangle] at (0,0) {};
                    \node[rectangle,draw,text=black, fill=\colorStepOne,opacity=\otherElementOpacity, minimum width=\minWidthOverview, minimum height=\minHeightOverview, align=center] (r1) at (\xStartOverview,\yStartOverview - \stepSizeOverview * 0) {Hintergrundrauschen};
                    \node[rectangle,draw,text=black, fill=\colorStepTwo,opacity=\otherElementOpacity, minimum width=\minWidthOverview, minimum height=\minHeightOverview, align=center] (r2) at (\xStartOverview,\yStartOverview - \stepSizeOverview * 1) {Falsche Peaks};
                    \node[rectangle,draw,text=black, fill=\colorStepThree, opacity=\otherElementOpacity,minimum width=\minWidthOverview, minimum height=\minHeightOverview, align=center] (r3) at (\xStartOverview,\yStartOverview - \stepSizeOverview * 2) {Irrelevante Peaks};
                    \node[rectangle,draw,text=black, fill=\colorStepFour, opacity=\currElementOpacity,minimum width=\minWidthOverview, minimum height=\minHeightOverview, align=center] (r4) at (\xStartOverview,\yStartOverview - \stepSizeOverview * 3) {$+1$ Normierung};
                    \node[rectangle,draw,text=black, fill=\colorStepFive, opacity=\otherElementOpacity,minimum width=\minWidthOverview, minimum height=\minHeightOverview, align=center] (r5) at (\xStartOverview,\yStartOverview - \stepSizeOverview * 4) {Zusammenfassen};
                    \draw[->, ultra thick] (r1) -- (r2);
                    \draw[->, ultra thick] (r2) -- (r3);
                    \draw[->, ultra thick] (r3) -- (r4);
                    \draw[->, ultra thick] (r4) -- (r5);
                \end{tikzpicture}
            \end{minipage}
        }%
        \onslide<2-> {
            \begin{minipage}[t]{.45\linewidth}
                \centering
                \begin{tikzpicture}[scale=\tikzScale, baseline=(current bounding box.center)]
                    % Alte Werte
                    % %\draw[thick] (0.382, 0.0) -- (0.382, {ln(1.7)});
                    % %\draw[thick] (0.476, 0.0) -- (0.476, {ln(2.7)});
                    % %\draw[thick] (0.456, 0.0) -- (0.456, {ln(1.8)});
                    % %\draw[thick] (0.6859999999999999, 0.0) -- (0.6859999999999999, {ln(2.7)});
                    % %\draw[thick] (0.6839999999999999, 0.0) -- (0.6839999999999999, {ln(1.8)});
                    % \draw[thick] (1.452, 0.0) -- (1.452, {ln(1.3)});
                    % \draw[thick] (1.426, 0.0) -- (1.426, {ln(1.9)});
                    % \draw[thick] (1.548, 0.0) -- (1.548, {ln(1.9)});
                    % \draw[thick] (1.6740000000000002, 0.0) -- (1.6740000000000002, {ln(1.5)});
                    % \draw[thick] (1.788, 0.0) -- (1.788, {ln(2.5)});
                    % \draw[thick] (2.488, 0.0) -- (2.488, {ln(2.9)});
                    % \draw[thick] (2.504, 0.0) -- (2.504, {ln(2.0)});
                    % \draw[thick] (2.682, 0.0) -- (2.682, {ln(2.0)});
                    % \draw[thick] (2.702, 0.0) -- (2.702, {ln(2.5)});
                    % \draw[thick] (3.46, 0.0) -- (3.46, {ln(2.3)});
                    % \draw[thick] (3.516, 0.0) -- (3.516, {ln(1.1)});
                    % \draw[thick] (3.584, 0.0) -- (3.584, {ln(1.8)});
                    % \draw[thick] (3.652, 0.0) -- (3.652, {ln(2.0)});
                    % \draw[thick] (3.838, 0.0) -- (3.838, {ln(1.5)});
                    % \draw[thick] (3.8819999999999997, 0.0) -- (3.8819999999999997, {ln(2.6)});
                    % %\draw[thick] (4.38, 0.0) -- (4.38, {ln(1.1)});
                    % %\draw[thick] (4.456, 0.0) -- (4.456, {ln(2.2)});
                    % %\draw[thick] (4.644, 0.0) -- (4.644, {ln(2.6)});
                    % %\draw[thick] (4.675999999999999, 0.0) -- (4.675999999999999, {ln(2.5)});
                    % %\draw[thick] (4.898000000000001, 0.0) -- (4.898000000000001, {ln(1.2)});

                    \draw[color=\colorPeakWithChargeDistance,thick] (1.452, 0.0) -- (1.452, {ln(1.3)});
                    \draw[color=\colorPeakWithChargeDistance,thick] (1.426, 0.0) -- (1.426, {ln(1.9)});
                    \draw[color=\colorPeakWithChargeDistance,thick] (1.548, 0.0) -- (1.548, {ln(1.9)});
                    \draw[color=\colorPeakWithChargeDistance,thick] (1.6740000000000002, 0.0) -- (1.6740000000000002, {ln(1.5)});
                    \draw[color=\colorMonoisotopicMassPeak,thick] (1.788, 0.0) -- (1.788, {ln(2.5)});

                    \draw[color=\colorPeakWithChargeDistance,thick] (2.488, 0.0) -- (2.488, {ln(2.9)});
                    \draw[color=\colorPeakWithChargeDistance,thick] (2.504, 0.0) -- (2.504, {ln(2.0)});
                    \draw[color=\colorPeakWithChargeDistance,thick] (2.682, 0.0) -- (2.682, {ln(2.0)});
                    \draw[color=\colorMonoisotopicMassPeak,thick] (2.702, 0.0) -- (2.702, {ln(2.5)});

                    \draw[color=\colorPeakWithChargeDistance,thick] (3.46, 0.0) -- (3.46, {ln(2.3)});
                    \draw[color=\colorPeakWithChargeDistance,thick] (3.516, 0.0) -- (3.516, {ln(1.1)});
                    \draw[color=\colorPeakWithChargeDistance,thick] (3.584, 0.0) -- (3.584, {ln(1.8)});
                    \draw[color=\colorPeakWithChargeDistance,thick] (3.652, 0.0) -- (3.652, {ln(2.0)});
                    \draw[color=\colorPeakWithChargeDistance,thick] (3.838, 0.0) -- (3.838, {ln(1.5)});
                    \draw[color=\colorMonoisotopicMassPeak,thick] (3.8819999999999997, 0.0) -- (3.8819999999999997, {ln(2.6)});

                    \draw [<->,thick] (0,\yAxisHeight) node (yaxis) [above] {\yAxisUnit} |- (\xAxisLength,0) node (xaxis) [right] {\xAxisUnit};
                \end{tikzpicture}%
            \end{minipage}%
        }%
        \onslide<3-> {
            \textbf{$\rightarrow$}
        }%
        \onslide<3->{
            \begin{minipage}[t]{.45\linewidth}
                \centering
                \begin{tikzpicture}[scale=\tikzScale, baseline=(current bounding box.center)]
                    \draw[color=\colorPeakWithChargeDistance,thick] (1.452 + \normierungOffset, 0.0) -- (1.452 + \normierungOffset, {ln(1.3)});
                    \draw[color=\colorPeakWithChargeDistance,thick] (1.426 + \normierungOffset, 0.0) -- (1.426 + \normierungOffset, {ln(1.9)});
                    \draw[color=\colorPeakWithChargeDistance,thick] (1.548 + \normierungOffset, 0.0) -- (1.548 + \normierungOffset, {ln(1.9)});
                    \draw[color=\colorPeakWithChargeDistance,thick] (1.6740000000000002 + \normierungOffset, 0.0) -- (1.6740000000000002 + \normierungOffset, {ln(1.5)});
                    \draw[color=\colorMonoisotopicMassPeak,thick] (1.788, 0.0) -- (1.788, {ln(2.5)});

                    \draw[color=\colorPeakWithChargeDistance,thick] (2.488 + \normierungOffset, 0.0) -- (2.488 + \normierungOffset, {ln(2.9)});
                    \draw[color=\colorPeakWithChargeDistance,thick] (2.504 + \normierungOffset, 0.0) -- (2.504 + \normierungOffset, {ln(2.0)});
                    \draw[color=\colorPeakWithChargeDistance,thick] (2.682 + \normierungOffset, 0.0) -- (2.682 + \normierungOffset, {ln(2.0)});
                    \draw[color=\colorMonoisotopicMassPeak,thick] (2.702, 0.0) -- (2.702, {ln(2.5)});

                    \draw[color=\colorPeakWithChargeDistance,thick] (3.46 + \normierungOffset, 0.0) -- (3.46 + \normierungOffset, {ln(2.3)});
                    \draw[color=\colorPeakWithChargeDistance,thick] (3.516 + \normierungOffset, 0.0) -- (3.516 + \normierungOffset, {ln(1.1)});
                    \draw[color=\colorPeakWithChargeDistance,thick] (3.584 + \normierungOffset, 0.0) -- (3.584 + \normierungOffset, {ln(1.8)});
                    \draw[color=\colorPeakWithChargeDistance,thick] (3.652 + \normierungOffset, 0.0) -- (3.652 + \normierungOffset, {ln(2.0)});
                    \draw[color=\colorPeakWithChargeDistance,thick] (3.838 + \normierungOffset, 0.0) -- (3.838 + \normierungOffset, {ln(1.5)});
                    \draw[color=\colorMonoisotopicMassPeak,thick] (3.8819999999999997, 0.0) -- (3.8819999999999997, {ln(2.6)});

                    \draw [<->,thick] (0,\yAxisHeight) node (yaxis) [above] {\yAxisUnit} |- (\xAxisLength,0) node (xaxis) [right] {\xAxisUnit};
                \end{tikzpicture}%
            \end{minipage}%
        }%
    \end{frame}
    %%%%% %%%%% %%%%% END +1 Normierung Grafik %%%%% %%%%% %%%%%

    %%%%% %%%%% %%%%% BEGIN Merge Grafik %%%%% %%%%% %%%%%
    \newcommand{\mergeColorOne}{orange}
    \newcommand{\mergeColorTwo}{violet}
    \newcommand{\mergeColorThree}{green!70!}
    \newcommand{\mergeColorFour}{cyan}
    \newcommand{\mergeColorNotMerged}{black}
    \begin{frame}{pNovo+ Algorithmus \dashAndSpace Datenaufbereitung}
        \onslide<1->{
            \begin{minipage}[c]{0.775\textwidth}
                \centering
                \begin{itemize}
                    \item<1-> Zusammenfassen von Peaks
                    \item<1-> Abstand einen \textcolor{\highlightColor}{Schwellwert} unterschreitet
                    \item<1-> $y=$ \textcolor{\highlightColor}{Median} aus zusammengefassten Peaks
                \end{itemize}
            \end{minipage}
        }%
        \onslide<1->{
            \begin{minipage}[c]{0.2\textwidth}
                \centering
                \begin{tikzpicture}[font=\tiny]
                    %\node[rectangle] at (0,0) {};
                    \node[rectangle,draw,text=black, fill=\colorStepOne,opacity=\otherElementOpacity, minimum width=\minWidthOverview, minimum height=\minHeightOverview, align=center] (r1) at (\xStartOverview,\yStartOverview - \stepSizeOverview * 0) {Hintergrundrauschen};
                    \node[rectangle,draw,text=black, fill=\colorStepTwo,opacity=\otherElementOpacity, minimum width=\minWidthOverview, minimum height=\minHeightOverview, align=center] (r2) at (\xStartOverview,\yStartOverview - \stepSizeOverview * 1) {Falsche Peaks};
                    \node[rectangle,draw,text=black, fill=\colorStepThree, opacity=\otherElementOpacity,minimum width=\minWidthOverview, minimum height=\minHeightOverview, align=center] (r3) at (\xStartOverview,\yStartOverview - \stepSizeOverview * 2) {Irrelevante Peaks};
                    \node[rectangle,draw,text=black, fill=\colorStepFour, opacity=\otherElementOpacity,minimum width=\minWidthOverview, minimum height=\minHeightOverview, align=center] (r4) at (\xStartOverview,\yStartOverview - \stepSizeOverview * 3) {$+1$ Normierung};
                    \node[rectangle,draw,text=black, fill=\colorStepFive, opacity=\currElementOpacity,minimum width=\minWidthOverview, minimum height=\minHeightOverview, align=center] (r5) at (\xStartOverview,\yStartOverview - \stepSizeOverview * 4) {Zusammenfassen};
                    \draw[->, ultra thick] (r1) -- (r2);
                    \draw[->, ultra thick] (r2) -- (r3);
                    \draw[->, ultra thick] (r3) -- (r4);
                    \draw[->, ultra thick] (r4) -- (r5);
                \end{tikzpicture}
            \end{minipage}
        }%
        \onslide<2-> {
            \begin{minipage}[t]{.45\linewidth}
                \centering
                \begin{tikzpicture}[scale=\tikzScale, baseline=(current bounding box.center)]
                    % % \draw[thick] (0.382, 0.0) -- (0.382, {ln(1.7)});
                    % % \draw[thick] (0.476, 0.0) -- (0.476, {ln(2.7)});
                    % % \draw[thick] (0.456, 0.0) -- (0.456, {ln(1.8)});
                    % % \draw[thick] (0.6859999999999999, 0.0) -- (0.6859999999999999, {ln(2.7)});
                    % % \draw[thick] (0.6839999999999999, 0.0) -- (0.6839999999999999, {ln(1.8)});
                    % \draw[color=red,thick] (1.452, 0.0) -- (1.452, {ln(1.3)});
                    % \draw[color=red,thick] (1.426, 0.0) -- (1.426, {ln(1.9)});
                    % \draw[thick] (1.548, 0.0) -- (1.548, {ln(1.9)});
                    % \draw[thick] (1.6740000000000002, 0.0) -- (1.6740000000000002, {ln(1.5)});
                    % \draw[thick] (1.788, 0.0) -- (1.788, {ln(2.5)});
                    % \draw[color=red,thick] (2.488, 0.0) -- (2.488, {ln(2.9)});
                    % \draw[color=red,thick] (2.504, 0.0) -- (2.504, {ln(2.0)});
                    % \draw[color=red,thick] (2.682, 0.0) -- (2.682, {ln(2.0)});
                    % \draw[color=red,thick] (2.702, 0.0) -- (2.702, {ln(2.5)});
                    % \draw[thick] (3.46, 0.0) -- (3.46, {ln(2.3)});
                    % \draw[thick] (3.516, 0.0) -- (3.516, {ln(1.1)});
                    % \draw[thick] (3.584, 0.0) -- (3.584, {ln(1.8)});
                    % \draw[thick] (3.652, 0.0) -- (3.652, {ln(2.0)});
                    % \draw[color=red,thick] (3.838, 0.0) -- (3.838, {ln(1.5)});
                    % \draw[color=red,thick] (3.8819999999999997, 0.0) -- (3.8819999999999997, {ln(2.6)});
                    % % \draw[thick] (4.38, 0.0) -- (4.38, {ln(1.1)});
                    % % \draw[thick] (4.456, 0.0) -- (4.456, {ln(2.2)});
                    % % \draw[thick] (4.644, 0.0) -- (4.644, {ln(2.6)});
                    % % \draw[thick] (4.675999999999999, 0.0) -- (4.675999999999999, {ln(2.5)});
                    % % \draw[thick] (4.898000000000001, 0.0) -- (4.898000000000001, {ln(1.2)});

                    \draw[color=\mergeColorOne,thick] (1.452 + \normierungOffset, 0.0) -- (1.452 + \normierungOffset, {ln(1.3)});
                    \draw[color=\mergeColorOne,thick] (1.426 + \normierungOffset, 0.0) -- (1.426 + \normierungOffset, {ln(1.9)});
                    \draw[color=\mergeColorTwo,thick] (1.548 + \normierungOffset, 0.0) -- (1.548 + \normierungOffset, {ln(1.9)});
                    \draw[color=\mergeColorTwo,thick] (1.6740000000000002 + \normierungOffset, 0.0) -- (1.6740000000000002 + \normierungOffset, {ln(1.5)});
                    \draw[color=\mergeColorOne,thick] (1.788, 0.0) -- (1.788, {ln(2.5)});

                    \draw[color=\mergeColorThree,thick] (2.488 + \normierungOffset, 0.0) -- (2.488 + \normierungOffset, {ln(2.9)});
                    \draw[color=\mergeColorThree,thick] (2.504 + \normierungOffset, 0.0) -- (2.504 + \normierungOffset, {ln(2.0)});
                    \draw[color=\mergeColorNotMerged,thick] (2.682 + \normierungOffset, 0.0) -- (2.682 + \normierungOffset, {ln(2.0)});
                    \draw[color=\mergeColorNotMerged,thick] (2.702, 0.0) -- (2.702, {ln(2.5)});

                    \draw[color=\mergeColorFour,thick] (3.46 + \normierungOffset, 0.0) -- (3.46 + \normierungOffset, {ln(2.3)});
                    \draw[color=\mergeColorFour,thick] (3.516 + \normierungOffset, 0.0) -- (3.516 + \normierungOffset, {ln(1.1)});
                    \draw[color=\mergeColorFour,thick] (3.584 + \normierungOffset, 0.0) -- (3.584 + \normierungOffset, {ln(1.8)});
                    \draw[color=\mergeColorFour,thick] (3.652 + \normierungOffset, 0.0) -- (3.652 + \normierungOffset, {ln(2.0)});
                    \draw[color=\mergeColorNotMerged,thick] (3.838 + \normierungOffset, 0.0) -- (3.838 + \normierungOffset, {ln(1.5)});
                    \draw[color=\mergeColorFour,thick] (3.8819999999999997, 0.0) -- (3.8819999999999997, {ln(2.6)});

                    \draw [<->,thick] (0,\yAxisHeight) node (yaxis) [above] {\yAxisUnit} |- (\xAxisLength,0) node (xaxis) [right] {\xAxisUnit};
                \end{tikzpicture}%
            \end{minipage}%
        }%
        \onslide<3-> {
            \textbf{$\rightarrow$}
        }%
        \onslide<3-> {
            \begin{minipage}[t]{.45\linewidth}
                \centering
                \begin{tikzpicture}[scale=\tikzScale, baseline=(current bounding box.center)]
                    % %\draw[color=red,thick] (1.452, 0.0) -- (1.452, {ln(1.3)});
                    % %\draw[color=red,thick] (1.426, 0.0) -- (1.426, {ln(1.9)});
                    % \draw[color=red,ultra thick] ({(1.452+1.426)/2}, 0.0) -- ({(1.452+1.426)/2}, {(ln(1.3)+ln(1.9))/2});
                    %
                    % \draw[thick] (1.548, 0.0) -- (1.548, {ln(1.9)});
                    % \draw[thick] (1.6740000000000002, 0.0) -- (1.6740000000000002, {ln(1.5)});
                    % \draw[thick] (1.788, 0.0) -- (1.788, {ln(2.5)});
                    %
                    % %\draw[color=red,thick] (2.488, 0.0) -- (2.488, {ln(2.9)});
                    % %\draw[color=red,thick] (2.504, 0.0) -- (2.504, {ln(2.0)});
                    % \draw[color=red,ultra thick] ({(2.488+2.504)/2}, 0.0) -- ({(2.488+2.504)/2}, {(ln(2.9)+ln(2.0))/2});
                    %
                    % %\draw[color=red,thick] (2.682, 0.0) -- (2.682, {ln(2.0)});
                    % %\draw[color=red,thick] (2.702, 0.0) -- (2.702, {ln(2.5)});
                    % \draw[color=red,ultra thick] ({(2.682+2.702)/2}, 0.0) -- ({(2.682+2.702)/2}, {(ln(2.0+ln(2.5))/2});
                    %
                    % \draw[thick] (3.46, 0.0) -- (3.46, {ln(2.3)});
                    % \draw[thick] (3.516, 0.0) -- (3.516, {ln(1.1)});
                    % \draw[thick] (3.584, 0.0) -- (3.584, {ln(1.8)});
                    % \draw[thick] (3.652, 0.0) -- (3.652, {ln(2.0)});
                    %
                    % %\draw[color=red,thick] (3.838, 0.0) -- (3.838, {ln(1.5)});
                    % %\draw[color=red,thick] (3.8819999999999997, 0.0) -- (3.8819999999999997,{ln(2.6)});
                    % \draw[color=red,ultra thick] ({(3.838+3.8819999999999997)/2}, 0.0) -- ({(3.838+3.8819999999999997)/2}, {(ln(1.5)+ln(2.6))/2});

                    %       \draw[color=\colorPeakWithChargeDistance,thick] (1.452 + \normierungOffset, 0.0) -- (1.452 + \normierungOffset, {ln(1.3)});
                    % \draw[color=\colorPeakWithChargeDistance,thick] (1.426 + \normierungOffset, 0.0) -- (1.426 + \normierungOffset, {ln(1.9)});
                    \draw[color=\mergeColorOne,ultra thick] ({(1.426 + 1.452 + \normierungOffset * 2) / 2}, 0.0) -- ({(1.426 + 1.452 + \normierungOffset * 2) / 2}, {(ln(1.3) + ln(1.9)) / 2});
                    % \draw[color=\colorPeakWithChargeDistance,thick] (1.548 + \normierungOffset, 0.0) -- (1.548 + \normierungOffset, {ln(1.9)});
                    % \draw[color=\colorPeakWithChargeDistance,thick] (1.6740000000000002 + \normierungOffset, 0.0) -- (1.6740000000000002 + \normierungOffset, {ln(1.5)});
                    % \draw[color=\colorMonoisotopicMassPeak,thick] (1.788, 0.0) -- (1.788, {ln(2.5)});
                    \draw[color=\mergeColorTwo,ultra thick] ({(1.548 + 1.6740000000000002 + 1.788 + \normierungOffset * 3) / 3}, 0.0) -- ({(1.548 + 1.6740000000000002 + 1.788 + \normierungOffset * 3) / 3}, {(ln(1.9) + ln(1.5) + ln(2.5)) / 3});


                    % \draw[color=\colorPeakWithChargeDistance,thick] (2.488 + \normierungOffset, 0.0) -- (2.488 + \normierungOffset, {ln(2.9)});
                    % \draw[color=\colorPeakWithChargeDistance,thick] (2.504 + \normierungOffset, 0.0) -- (2.504 + \normierungOffset, {ln(2.0)});
                    \draw[color=\mergeColorThree,ultra thick] ({2.488 + 2.504 + \normierungOffset * 2) / 2}, 0.0) -- ({2.488 + 2.504 + \normierungOffset * 2) / 2}, {(ln(2.9) + ln(2.0)) / 2});
                    \draw[color=\mergeColorNotMerged,thick] (2.682 + \normierungOffset, 0.0) -- (2.682 + \normierungOffset, {ln(2.0)});
                    \draw[color=\mergeColorNotMerged,thick] (2.702, 0.0) -- (2.702, {ln(2.5)});

                    % \draw[color=black,thick] (3.46 + \normierungOffset, 0.0) -- (3.46 + \normierungOffset, {ln(2.3)});
                    % \draw[color=\colorPeakWithChargeDistance,thick] (3.516 + \normierungOffset, 0.0) -- (3.516 + \normierungOffset, {ln(1.1)});
                    % \draw[color=\colorPeakWithChargeDistance,thick] (3.584 + \normierungOffset, 0.0) -- (3.584 + \normierungOffset, {ln(1.8)});
                    % \draw[color=\colorPeakWithChargeDistance,thick] (3.652 + \normierungOffset, 0.0) -- (3.652 + \normierungOffset, {ln(2.0)});
                    \draw[color=\mergeColorFour,ultra thick] ({(3.46 + 3.516 + 3.584 + 3.652 + 3.8819999999999997 + \normierungOffset * 5) / 5}, 0.0) -- ({(3.46 + 3.516 + 3.584 + 3.652 + 3.8819999999999997 + \normierungOffset * 5) / 5}, {(ln(2.3) + ln(1.1) + ln(1.8) + ln(2.0) + ln(1.5)) / 5});
                    \draw[color=\mergeColorNotMerged,thick] (3.838 + \normierungOffset, 0.0) -- (3.838 + \normierungOffset, {ln(1.5)});
                    % \draw[color=\colorMonoisotopicMassPeak,thick] (3.8819999999999997, 0.0) -- (3.8819999999999997, {ln(2.6)});

                    \draw [<->,thick] (0,\yAxisHeight) node (yaxis) [above] {\yAxisUnit} |- (\xAxisLength,0) node (xaxis) [right] {\xAxisUnit};
                \end{tikzpicture}
            \end{minipage}
        }%
    \end{frame}
     %%%%% %%%%% %%%%% END Merge Grafik %%%%% %%%%% %%%%%


    \begin{frame}{pNovo+ Algorithmus \dashAndSpace Bildung eines Spektrumsgraphen}
        \onslide<1->{
            \begin{minipage}[c]{0.2\textwidth}
                \centering
                \includegraphics[scale=0.1]{./Presentation_Images/MassSpectraIcon.png}

                \vspace*{-0.4cm}
                $\Downarrow$
                \vspace*{0.2cm}

                \begin{tikzpicture}[scale=0.9]
                    \node [circle,draw,ultra thick,color=blue] (0) at (-0.75, 0.75) {};
                    \node [circle,draw,ultra thick,color=blue] (1) at (0.25, 1.5) {};
                    \node [circle,draw,ultra thick,color=blue] (2) at (-0.25, -0.25) {};
                    \node [circle,draw,ultra thick,color=blue] (3) at (0.75, 0.5) {};
                    \node [circle,draw,ultra thick,color=blue] (4) at (-1.25, -0.25) {};

                    \draw[->,thick] (4) to (0);
                    \draw[->,thick] (0) to (2);
                    \draw[->,thick] (2) to (3);
                    \draw[->,thick] (0) to (1);
                    \draw[->,thick] (0) to (3);
                    \draw[->,thick] (1) to (3);
                \end{tikzpicture}
            \end{minipage}
        }%
        \onslide<1->{
            \begin{minipage}[c]{0.775\textwidth}
                \begin{itemize}
                    \item<1-> Verwendung vorverarbeiteter MS2 Spektren
                    \item<1-> Peaks $\rightarrow$ Knoten:
                    \begin{itemize}
                        \item<2-> \textcolor{\highlightColor}{Peaks} $\widehat{=}$ \textcolor{\highlightColor}{Knotenpaar}
                        \item<2-> Knoten bekommen eine \gerquot{Masse}
                        \item<2-> Masse $\widehat{=}$ \massCharge Wert
                        \item<2-> Startknoten (Masse = 0)
                        \item<2-> Endknoten (Masse = Hauptpeak - \emph{M}(\ch{H2O}))
                    \end{itemize}
                \end{itemize}
            \end{minipage}
        }%
    \end{frame}

    \begin{frame}{pNovo+ Algorithmus \dashAndSpace Bildung eines Spektrumsgraphen}
        \onslide<1->{
            \begin{minipage}[c]{0.2\textwidth}
                \centering
                \includegraphics[scale=0.1]{./Presentation_Images/MassSpectraIcon.png}

                \vspace*{-0.4cm}
                $\Downarrow$
                \vspace*{0.2cm}

                \begin{tikzpicture}[scale=0.9]
                    \node [circle,draw,thick] (0) at (-0.75, 0.75) {};
                    \node [circle,draw,thick] (1) at (0.25, 1.5) {};
                    \node [circle,draw,thick] (2) at (-0.25, -0.25) {};
                    \node [circle,draw,thick] (3) at (0.75, 0.5) {};
                    \node [circle,draw,thick] (4) at (-1.25, -0.25) {};

                    \draw[->,ultra thick,color=blue] (4) to (0);
                    \draw[->,ultra thick,color=blue] (0) to (2);
                    \draw[->,ultra thick,color=blue] (2) to (3);
                    \draw[->,ultra thick,color=blue] (0) to (1);
                    \draw[->,ultra thick,color=blue] (0) to (3);
                    \draw[->,ultra thick,color=blue] (1) to (3);
                \end{tikzpicture}
            \end{minipage}
        }%
        \onslide<1->{
            \begin{minipage}[c]{0.775\textwidth}
                \begin{itemize}
                    \item<1-> Gerichtete Kanten zwischen Knotenpaar, wenn:
                    \begin{itemize}
                        \item<2-> Massendifferenz genau Masse \textcolor{\highlightColor}{einer} AA entspricht
                        \item<2-> Massendifferenz genau Masse \textcolor{\highlightColor}{zwei} AA entsprechen
                    \end{itemize}
                    \item<3-> $N + \binom{n+N-1}{N-1}$ Differenzen ($n=2$, $N=20$)
                    \item<3-> \textcolor{\highlightColor}{$230$} Differenzen
                \end{itemize}
            \end{minipage}
        }%
    \end{frame}


    \begin{frame}{pNovo+ Algorithmus \dashAndSpace Bildung eines Spektrumsgraphen}
        \begin{itemize}
            \item<1-> Ergebnis: Directed acyclic graph (DAG)
            \item<2-> Beispiel eines Spektrumsgraphen:
        \end{itemize}
        \onslide<2->{
            \centering
            % Exemplarischer Spektrumsgraphen
            \begin{tikzpicture}[scale=.7,auto=left,every node/.style={circle,fill=blue!20}]
                \node (begin) at (0,10) {Start};
                \node (end) at (12,10) {End};
                \node (n1) at (2,9) {A};
                \node (n1b) at (2,11) {A};
                \node (n2) at (4,9) {B};
                \node (n2b) at (4,11) {B};
                \node (n3) at (6,9) {C};
                \node (n3b) at (6,11) {C};
                \node (n4) at (8,9) {D};
                \node (n4b) at (8,11) {D};
                \node (n5) at (10,9) {E};
                \node (n5b) at (10,11) {E};

                \tikzset{every node/.style={}}
                \draw[->] (begin) to node[above] {} (n1);
                \draw[->] (n1b) to node[above] {} (n2);
                \draw[->] (n1) to node[above] {} (n2);
                \draw[->] (n1) to[bend right] node[above] {} (n3);
                \draw[->] (n4b) to node[above] {} (n5);

                %\draw[->] (n2) to node[above] {} (n3b);
                \draw[->] (n2) to node[above] {} (n4b);
                \draw[->] (n2) to[bend left] node[above] {} (n4);
                \draw[->] (n4) to node[above] {} (n5);
                %\draw[->] (n1) to[bend right] node[above] {} (n5);
                \draw[->] (n5) to  node[above] {} (end);
                \draw[->] (n5b) to node[above] {} (end);

                %   \foreach \from/\to in {begin/n1,n5/end}
                %   {
                %   \tikzset{every node/.style={}}
                %     \draw[->] (\from) -- node[above] {} (\to);
                %     }
                %     \foreach \from/\to in {n1/n2,n4/n5,n3/n4,n1/n4,n1b/n2}
                %   {
                %   \tikzset{every node/.style={}}
                %     \draw[->] (\from) to[bend left] node[above] {} (\to);
                %     }
                %     \foreach \from/\to in {n1/n3,n3/n4b,n3/n4,n4/n5}
                %   {
                %   \tikzset{every node/.style={}}
                %     \draw[->] (\from) to[bend right] node[below] {} (\to);
                %     }
            \end{tikzpicture}
        }%
        \begin{itemize}
            \item<3-> Alle möglichen Pfade von Start nach End ermitteln
            \item<3-> Scoring Funktion \gerquot{bewertet} jeden Pfad
            \item<3-> Pfad mit dem \textcolor{\highlightColor}{höchsten Scoring Wert} ist das Ergebnis
        \end{itemize}
    \end{frame}

    \begin{frame}{pNovo+ Algorithmus \dashAndSpace Evaluierung}
        \onslide<1->{
            \begin{minipage}[c]{0.2\textwidth}
                \centering
                \includegraphics[scale=0.1]{./Presentation_Images/EvaluationIcon.png}
            \end{minipage}
        }%
        \onslide<1->{
            \begin{minipage}[c]{0.775\textwidth}
                \begin{itemize}
                    \item<1-> 8677 Datensätze
                    \item<1-> Erfolgreiche Sequenzierungen: \textcolor{brown!80!}{$81,2\%$}
                    \item<1-> Alternativalgorithmus (PEAKS): \textcolor{orange!75!}{$71.8\%$}
                \end{itemize}
                \vspace*{0.4cm}
                \begin{itemize}
                    \item<2-> pNovo+ \textcolor{\highlightColor}{besser als die Konkurrenz!}
                    \item<2-> Side Note: pNovo+ ist \textcolor{\highlightColor}{frei verfügbar}
                \end{itemize}
            \end{minipage}
        }%
    \end{frame}
    %%%%% %%%%% %%%%% %%%%% %%%%% BEGIN pNovo+ Algorithmus %%%%% %%%%% %%%%% %%%%% %%%%%



    %%%%% %%%%% %%%%% %%%%% %%%%% BEGIN Open-pNovo Algorithmus %%%%% %%%%% %%%%% %%%%% %%%%%
    \section{Open-pNovo Algorithmus}
    \begin{frame}{Open-pNovo Algorithmus \dashAndSpace Hintergrund}
        \begin{itemize}
            \item<1-> Peptide sind nicht zwingend stabil
            \item<1-> Wechselwirkungen können die Sequenz abändern
            \item<1-> Posttranslationale Proteinmodifikationenen (PTM)
            % Beispiel einer Wechselwirkung zeigen
            \item<2-> Mit De-Novo-Algorithmen an sich kein Problem ...
            \item<3-> ... wenn nach der Änderung eine AA zurückbleibt
        \end{itemize}
    \end{frame}

    \begin{frame}{Open-pNovo Algorithmus \dashAndSpace PTM}
        \begin{itemize}
            \item<1-> Bildung von nicht proteinogenen AA möglich
            \item<1-> AA, die normalerweise nicht in Peptiden vorkommen
            \item<2-> Spektrogramm zeigt solche AA
            \item<2-> Änderungen können von pNovo+ nicht erkannt werden
        \end{itemize}
        \begin{itemize}
            \item<3-> $\Rightarrow$ pNovo+ erzeugt zwangsweise Fehler
        \end{itemize}
    \end{frame}

    \begin{frame}{Open-pNovo Algorithmus \dashAndSpace RankBoost}
        \begin{itemize}
            \item<1-> Neue Scoring Funktion: RankBoost
            \item<1-> Machine Learning Algorithmus aus 2003
            \item<1-> Erweiterung des AdaBoost Algorithmus
            \item<1-> Präferenzen in Datensätzen zu erkennen
        \end{itemize}
        \begin{itemize}
            \item<2-> Filterung der nicht gültigen AA
        \end{itemize}
    \end{frame}

    \begin{frame}{Open-pNovo Algorithmus \dashAndSpace Evaluierung von Open-pNovo}
        \onslide<1->{
            \begin{minipage}[c]{0.2\textwidth}
                \centering
                \includegraphics[scale=0.1]{./Presentation_Images/TestingIcon.png}
            \end{minipage}
        }%
        \onslide<1->{
            \begin{minipage}[c]{0.775\textwidth}
                \begin{itemize}
                    \item<1-> 45450 Datensätze
                    \item<2-> Erfolgreiche Sequenzierungen: \textcolor{brown!80!}{$76,3\%$}
                    \item<2-> pNovo+: \textcolor{brown!80!}{$74,5\%$}
                    \item<2-> Alternativalgorithmus PEAKS: \textcolor{orange!75!}{$73,1\%$}
                    \item<2-> 2. Alternativalgorithmus Novor: \textcolor{orange!75!}{$39,9\%$}
                \end{itemize}
                \vspace*{0.3cm}
                \begin{itemize}
                    \item<3-> Verbesserung im Vergleich zu pNovo+
                    \item<4-> Allerdings: Weniger als $2\%$ Punkte
                \end{itemize}
            \end{minipage}
        }%
    \end{frame}
    %%%%% %%%%% %%%%% %%%%% %%%%% END Open-pNovo Algorithmus %%%%% %%%%% %%%%% %%%%% %%%%%



    %%%%% %%%%% %%%%% %%%%% %%%%% BEGIN Zusammenfassung %%%%% %%%%% %%%%% %%%%% %%%%%
    \section{Zusammenfassung}
    \begin{frame}{Zusammenfassung}
        \begin{itemize}
            \item<1-> De-Novo-Sequenzierung leichter durchführbar
            \item<1-> Beide Algorithmen liefern \textcolor{\highlightColor}{sehr gute} Ergebnisse
            \item<1-> pNovo+ Ansatz mit Spektrumsgraphen ist wirkungsvoll
            \item<1-> Open-pNovo erkennt zuverlässig Proben mit PTMs
            \vspace*{0.5cm}
            \item<2-> Dennoch: hoher Optimierungsbedarf besteht weiterhin
        \end{itemize}
    \end{frame}
    %%%%% %%%%% %%%%% %%%%% %%%%% END Zusammenfassung %%%%% %%%%% %%%%% %%%%% %%%%%



    %%%%% %%%%% %%%%% %%%%% %%%%% BEGIN Ende %%%%% %%%%% %%%%% %%%%% %%%%%
    \begin{frame}{The End}
        \onslide<1->{
            \begin{flushleft}
                \begin{large}
                    \texttt{Danke für die Aufmerksamkeit :)}
                \end{large}
            \end{flushleft}
            \vspace*{0.2cm}
            \begin{flushright}
                \begin{Large}
                    \texttt{Fragen?}
                \end{Large}
            \end{flushright}
        }%
        \onslide<2->{
            \begin{center}
                \includegraphics[width=0.03\linewidth, angle=-15]{./Presentation_Images/vecteezy_question-mark-with-3d-vector-icon-cartoon-minimal-style_16626300.png}
                \includegraphics[width=0.04\linewidth]{./Presentation_Images/vecteezy_question-mark-with-3d-vector-icon-cartoon-minimal-style_16626300.png}
                \includegraphics[width=0.03\linewidth, angle=+15]{./Presentation_Images/vecteezy_question-mark-with-3d-vector-icon-cartoon-minimal-style_16626300.png}
                \\
                \vspace*{0.10cm}
                \includegraphics[width=0.2\linewidth]{./Presentation_Images/NewTux.png}
        }%
        \end{center}
    \end{frame}
    %%%%% %%%%% %%%%% %%%%% %%%%% BEGIN Ende %%%%% %%%%% %%%%% %%%%% %%%%%

\end{document}
%%%%% %%%%% %%%%% %%%%% %%%%% %%%%% %%%%% %%%%%              %%%%% %%%%% %%%%% %%%%% %%%%% %%%%% %%%%% %%%%%
%%%%% %%%%% %%%%% %%%%% %%%%% %%%%% %%%%% %%%%% END document %%%%% %%%%% %%%%% %%%%% %%%%% %%%%% %%%%% %%%%%
%%%%% %%%%% %%%%% %%%%% %%%%% %%%%% %%%%% %%%%%              %%%%% %%%%% %%%%% %%%%% %%%%% %%%%% %%%%% %%%%%
