\documentclass[a4paper, 12pt]{article}
\usepackage[utf8]{inputenc} % UTF-8 Kodierung verwenden
\usepackage[backend=biber, sorting=none]{biblatex}
\usepackage[total={6.5in, 9in}]{geometry}
% \usepackage[onehalfspacing]{setspace} % 1.5 Spacing 
\usepackage[singlespacing]{setspace} % 1 Spacing 
\usepackage[T1]{fontenc}    % Fonts mit westeuropäischer Codierung verwenden
\usepackage[ngerman]{babel} % Neue deutsche Sprache
\usepackage{fancyhdr}       % Kopf- und Fusszeilen
\usepackage{tikz}           % Fuer das Erstellen von einfachen Grafiken
\usepackage{float}          % Fuer den Positionierungsbefehl '[H]'
\usepackage{fancyhdr}       % Angepasste Header und Footer
\usepackage{titling}        % Fuer Befehle wie \thetitle
% \usepackage{showframe}     % Boxen mit Rand visualisieren (nur für das Schreiben des Dokuments brauchbar!)
\usepackage{csquotes}
\usepackage{translator}
\usepackage[
nonumberlist, %keine Seitenzahlen anzeigen
%acronym,      %ein Abkürzungsverzeichnis erstellen
toc,          %Einträge im Inhaltsverzeichnis
section,      %im Inhaltsverzeichnis auf section-Ebene erscheinen
nopostdot     %Den Punkt am Ende jeder Beschreibung deaktivieren
]{glossaries}
\makenoidxglossaries
 
% \setlength{\abovecaptionskip}{1ex}
% \setlength{\belowcaptionskip}{1ex}
\setlength{\floatsep}{24pt}
\setlength{\textfloatsep}{24pt}
\setlength{\headheight}{15pt}

\setcounter{tocdepth}{1}

\title{De-Novo-Sequencing using Spectrum-Graphs, enabling Open Searches}
\author{Dominik Habermann}
\date{\today}

% Kopf- und Fussnoten anpassen
\pagestyle{fancy}
\fancyhf{}
\fancyhead[L]{\thetitle}
%\fancyhead[R]{\thetitle} 
\fancyfoot[C]{\thepage}



\addbibresource{P2_De-Novo-Sequencing using Spectrum-Graphs.bib}
\documentclass[a4paper, 12pt]{article}
\usepackage[utf8]{inputenc} % UTF-8 Kodierung verwenden
\usepackage[backend=biber, sorting=none]{biblatex}
\usepackage[total={6.5in, 9in}]{geometry}
% \usepackage[onehalfspacing]{setspace} % 1.5 Spacing 
\usepackage[singlespacing]{setspace} % 1 Spacing 
\usepackage[T1]{fontenc}    % Fonts mit westeuropäischer Codierung verwenden
\usepackage[ngerman]{babel} % Neue deutsche Sprache
\usepackage{fancyhdr}       % Kopf- und Fusszeilen
\usepackage{tikz}           % Fuer das Erstellen von einfachen Grafiken
\usepackage{float}          % Fuer den Positionierungsbefehl '[H]'
\usepackage{fancyhdr}       % Angepasste Header und Footer
\usepackage{titling}        % Fuer Befehle wie \thetitle
% \usepackage{showframe}     % Boxen mit Rand visualisieren (nur für das Schreiben des Dokuments brauchbar!)
\usepackage{csquotes}
\usepackage{translator}
\usepackage[
nonumberlist, %keine Seitenzahlen anzeigen
%acronym,      %ein Abkürzungsverzeichnis erstellen
toc,          %Einträge im Inhaltsverzeichnis
section,      %im Inhaltsverzeichnis auf section-Ebene erscheinen
nopostdot     %Den Punkt am Ende jeder Beschreibung deaktivieren
]{glossaries}
\makenoidxglossaries
 
% \setlength{\abovecaptionskip}{1ex}
% \setlength{\belowcaptionskip}{1ex}
\setlength{\floatsep}{24pt}
\setlength{\textfloatsep}{24pt}
\setlength{\headheight}{15pt}

\setcounter{tocdepth}{1}

\title{De-Novo-Sequencing using Spectrum-Graphs, enabling Open Searches}
\author{Dominik Habermann}
\date{\today}

% Kopf- und Fussnoten anpassen
\pagestyle{fancy}
\fancyhf{}
\fancyhead[L]{\thetitle}
%\fancyhead[R]{\thetitle} 
\fancyfoot[C]{\thepage}



\addbibresource{P2_De-Novo-Sequencing using Spectrum-Graphs.bib}
\documentclass[a4paper, 12pt]{article}
\usepackage[utf8]{inputenc} % UTF-8 Kodierung verwenden
\usepackage[backend=biber, sorting=none]{biblatex}
\usepackage[total={6.5in, 9in}]{geometry}
% \usepackage[onehalfspacing]{setspace} % 1.5 Spacing 
\usepackage[singlespacing]{setspace} % 1 Spacing 
\usepackage[T1]{fontenc}    % Fonts mit westeuropäischer Codierung verwenden
\usepackage[ngerman]{babel} % Neue deutsche Sprache
\usepackage{fancyhdr}       % Kopf- und Fusszeilen
\usepackage{tikz}           % Fuer das Erstellen von einfachen Grafiken
\usepackage{float}          % Fuer den Positionierungsbefehl '[H]'
\usepackage{fancyhdr}       % Angepasste Header und Footer
\usepackage{titling}        % Fuer Befehle wie \thetitle
% \usepackage{showframe}     % Boxen mit Rand visualisieren (nur für das Schreiben des Dokuments brauchbar!)
\usepackage{csquotes}
\usepackage{translator}
\usepackage[
nonumberlist, %keine Seitenzahlen anzeigen
%acronym,      %ein Abkürzungsverzeichnis erstellen
toc,          %Einträge im Inhaltsverzeichnis
section,      %im Inhaltsverzeichnis auf section-Ebene erscheinen
nopostdot     %Den Punkt am Ende jeder Beschreibung deaktivieren
]{glossaries}
\makenoidxglossaries
 
% \setlength{\abovecaptionskip}{1ex}
% \setlength{\belowcaptionskip}{1ex}
\setlength{\floatsep}{24pt}
\setlength{\textfloatsep}{24pt}
\setlength{\headheight}{15pt}

\setcounter{tocdepth}{1}

\title{De-Novo-Sequencing using Spectrum-Graphs, enabling Open Searches}
\author{Dominik Habermann}
\date{\today}

% Kopf- und Fussnoten anpassen
\pagestyle{fancy}
\fancyhf{}
\fancyhead[L]{\thetitle}
%\fancyhead[R]{\thetitle} 
\fancyfoot[C]{\thepage}



\addbibresource{P2_De-Novo-Sequencing using Spectrum-Graphs.bib}
\documentclass[a4paper, 12pt]{article}
\usepackage[utf8]{inputenc} % UTF-8 Kodierung verwenden
\usepackage[backend=biber, sorting=none]{biblatex}
\usepackage[total={6.5in, 9in}]{geometry}
% \usepackage[onehalfspacing]{setspace} % 1.5 Spacing 
\usepackage[singlespacing]{setspace} % 1 Spacing 
\usepackage[T1]{fontenc}    % Fonts mit westeuropäischer Codierung verwenden
\usepackage[ngerman]{babel} % Neue deutsche Sprache
\usepackage{fancyhdr}       % Kopf- und Fusszeilen
\usepackage{tikz}           % Fuer das Erstellen von einfachen Grafiken
\usepackage{float}          % Fuer den Positionierungsbefehl '[H]'
\usepackage{fancyhdr}       % Angepasste Header und Footer
\usepackage{titling}        % Fuer Befehle wie \thetitle
% \usepackage{showframe}     % Boxen mit Rand visualisieren (nur für das Schreiben des Dokuments brauchbar!)
\usepackage{csquotes}
\usepackage{translator}
\usepackage[
nonumberlist, %keine Seitenzahlen anzeigen
%acronym,      %ein Abkürzungsverzeichnis erstellen
toc,          %Einträge im Inhaltsverzeichnis
section,      %im Inhaltsverzeichnis auf section-Ebene erscheinen
nopostdot     %Den Punkt am Ende jeder Beschreibung deaktivieren
]{glossaries}
\makenoidxglossaries
 
% \setlength{\abovecaptionskip}{1ex}
% \setlength{\belowcaptionskip}{1ex}
\setlength{\floatsep}{24pt}
\setlength{\textfloatsep}{24pt}
\setlength{\headheight}{15pt}

\setcounter{tocdepth}{1}

\title{De-Novo-Sequencing using Spectrum-Graphs, enabling Open Searches}
\author{Dominik Habermann}
\date{\today}

% Kopf- und Fussnoten anpassen
\pagestyle{fancy}
\fancyhf{}
\fancyhead[L]{\thetitle}
%\fancyhead[R]{\thetitle} 
\fancyfoot[C]{\thepage}



\addbibresource{P2_De-Novo-Sequencing using Spectrum-Graphs.bib}
\input{P2_De-Novo-Sequencing using Spectrum-Graphs.gls}
\input{P2_De-Novo-Sequencing using Spectrum-Graphs.acr}

\newcommand{\gerquot}[1]{\glqq#1\grqq}
\newcommand{\dashAndSpace}{\textendash \space}

\renewcommand{\floatpagefraction}{0.8}
% Workaround um die Überschrift des Glossars anzupassen
% Siehe: https://tex.stackexchange.com/questions/426390/how-can-i-rename-the-header-titles-of-the-glossary
\addto\captionsngerman{%
\renewcommand*{\glossaryname}{Begriffserklärungen}%
}


 
%%%%% %%%%% %%%%% %%%%% %%%%% \begin{document} %%%%% %%%%% %%%%% %%%%% %%%%%
\begin{document}

\maketitle

\section{Einleitung}
Im ersten Kapitel findet zu Beginn eine Erklärung der wichtigsten Begriffe und Abkürzungen statt. Dazu wird eine Themenabgrenzung durchgeführt sowie die Ausgangssituation beschrieben.

\printnoidxglossaries

\subsection{Themenabgrenzung}
Folgende Aspekte sind Bestandteil dieser Ausarbeitung:
\begin{itemize}
   \item Was ist die\gls{gls:DeNovo}?
   \item Was erhofft man sich von dieser Technologie?
   \item Welche Probleme liegen vor, die von der Seite der Informatik gelöst / verbessert werden können?
   \item Inwiefern spielen die Spektrums-Graphen dabei eine Rolle?
\end{itemize}

\subsection{Ausgangssituation}
Mit Hilfe der\gls{gls:DeNovo} ist grundsätzlich die Bestimmung von unbekannten Aminosäuresequenzen möglich. Das Verfahren arbeitet allerdings nicht in jeder Situation zuverlässig genug. Dadurch wird das Ermitteln von unbekannten Sequenzen erschwert. Auch bei bereits bekannten Sequenzen führt die nicht ausreichende Zuverlässigkeit dazu, dass bei Ergebnissen nicht sicher unterschieden werden kann, ob eine Änderung in der Aminosäuresequenz vorliegt oder ob fehlerhafte Daten bestimmt wurden.
Das Ziel ist mit Unterstützung von Software eine effizente Möglichkeit bereitzustellen, um die Zuverlässigkeit der\gls{gls:DeNovo} zu erhöhen. 



\section{De-Novo-Peptidsequenzierung und Spektrums-Graphen im Detail}
\subsection{Datengewinnung}
\subsection{Datenauswertung}
 

\section{Ergebnisse/Evaluierung}
\subsection{Probleme in der Praxis}
\subsection{Lösungsansätze}


\section{Zusammenfassung}
\subsection{Ungelöste Probleme}
\subsection{Kritische Betrachtung}


\section{Section}
\cite{OpenPNovo}
\cite{pNovoPlus}

\begingroup
\setlength{\emergencystretch}{.5em}
\printbibliography
\endgroup

\appendix
\section{Appendix Section}
Appendix



\end{document}
%%%%% %%%%% %%%%% %%%%% %%%%% \end{document} %%%%% %%%%% %%%%% %%%%% %%%%%

\documentclass[a4paper, 12pt]{article}
\usepackage[utf8]{inputenc} % UTF-8 Kodierung verwenden
\usepackage[backend=biber, sorting=none]{biblatex}
\usepackage[total={6.5in, 9in}]{geometry}
% \usepackage[onehalfspacing]{setspace} % 1.5 Spacing 
\usepackage[singlespacing]{setspace} % 1 Spacing 
\usepackage[T1]{fontenc}    % Fonts mit westeuropäischer Codierung verwenden
\usepackage[ngerman]{babel} % Neue deutsche Sprache
\usepackage{fancyhdr}       % Kopf- und Fusszeilen
\usepackage{tikz}           % Fuer das Erstellen von einfachen Grafiken
\usepackage{float}          % Fuer den Positionierungsbefehl '[H]'
\usepackage{fancyhdr}       % Angepasste Header und Footer
\usepackage{titling}        % Fuer Befehle wie \thetitle
% \usepackage{showframe}     % Boxen mit Rand visualisieren (nur für das Schreiben des Dokuments brauchbar!)
\usepackage{csquotes}
\usepackage{translator}
\usepackage[
nonumberlist, %keine Seitenzahlen anzeigen
%acronym,      %ein Abkürzungsverzeichnis erstellen
toc,          %Einträge im Inhaltsverzeichnis
section,      %im Inhaltsverzeichnis auf section-Ebene erscheinen
nopostdot     %Den Punkt am Ende jeder Beschreibung deaktivieren
]{glossaries}
\makenoidxglossaries
 
% \setlength{\abovecaptionskip}{1ex}
% \setlength{\belowcaptionskip}{1ex}
\setlength{\floatsep}{24pt}
\setlength{\textfloatsep}{24pt}
\setlength{\headheight}{15pt}

\setcounter{tocdepth}{1}

\title{De-Novo-Sequencing using Spectrum-Graphs, enabling Open Searches}
\author{Dominik Habermann}
\date{\today}

% Kopf- und Fussnoten anpassen
\pagestyle{fancy}
\fancyhf{}
\fancyhead[L]{\thetitle}
%\fancyhead[R]{\thetitle} 
\fancyfoot[C]{\thepage}



\addbibresource{P2_De-Novo-Sequencing using Spectrum-Graphs.bib}
\input{P2_De-Novo-Sequencing using Spectrum-Graphs.gls}
\input{P2_De-Novo-Sequencing using Spectrum-Graphs.acr}

\newcommand{\gerquot}[1]{\glqq#1\grqq}
\newcommand{\dashAndSpace}{\textendash \space}

\renewcommand{\floatpagefraction}{0.8}
% Workaround um die Überschrift des Glossars anzupassen
% Siehe: https://tex.stackexchange.com/questions/426390/how-can-i-rename-the-header-titles-of-the-glossary
\addto\captionsngerman{%
\renewcommand*{\glossaryname}{Begriffserklärungen}%
}


 
%%%%% %%%%% %%%%% %%%%% %%%%% \begin{document} %%%%% %%%%% %%%%% %%%%% %%%%%
\begin{document}

\maketitle

\section{Einleitung}
Im ersten Kapitel findet zu Beginn eine Erklärung der wichtigsten Begriffe und Abkürzungen statt. Dazu wird eine Themenabgrenzung durchgeführt sowie die Ausgangssituation beschrieben.

\printnoidxglossaries

\subsection{Themenabgrenzung}
Folgende Aspekte sind Bestandteil dieser Ausarbeitung:
\begin{itemize}
   \item Was ist die\gls{gls:DeNovo}?
   \item Was erhofft man sich von dieser Technologie?
   \item Welche Probleme liegen vor, die von der Seite der Informatik gelöst / verbessert werden können?
   \item Inwiefern spielen die Spektrums-Graphen dabei eine Rolle?
\end{itemize}

\subsection{Ausgangssituation}
Mit Hilfe der\gls{gls:DeNovo} ist grundsätzlich die Bestimmung von unbekannten Aminosäuresequenzen möglich. Das Verfahren arbeitet allerdings nicht in jeder Situation zuverlässig genug. Dadurch wird das Ermitteln von unbekannten Sequenzen erschwert. Auch bei bereits bekannten Sequenzen führt die nicht ausreichende Zuverlässigkeit dazu, dass bei Ergebnissen nicht sicher unterschieden werden kann, ob eine Änderung in der Aminosäuresequenz vorliegt oder ob fehlerhafte Daten bestimmt wurden.
Das Ziel ist mit Unterstützung von Software eine effizente Möglichkeit bereitzustellen, um die Zuverlässigkeit der\gls{gls:DeNovo} zu erhöhen. 



\section{De-Novo-Peptidsequenzierung und Spektrums-Graphen im Detail}
\subsection{Datengewinnung}
\subsection{Datenauswertung}
 

\section{Ergebnisse/Evaluierung}
\subsection{Probleme in der Praxis}
\subsection{Lösungsansätze}


\section{Zusammenfassung}
\subsection{Ungelöste Probleme}
\subsection{Kritische Betrachtung}


\section{Section}
\cite{OpenPNovo}
\cite{pNovoPlus}

\begingroup
\setlength{\emergencystretch}{.5em}
\printbibliography
\endgroup

\appendix
\section{Appendix Section}
Appendix



\end{document}
%%%%% %%%%% %%%%% %%%%% %%%%% \end{document} %%%%% %%%%% %%%%% %%%%% %%%%%


\newcommand{\gerquot}[1]{\glqq#1\grqq}
\newcommand{\dashAndSpace}{\textendash \space}

\renewcommand{\floatpagefraction}{0.8}
% Workaround um die Überschrift des Glossars anzupassen
% Siehe: https://tex.stackexchange.com/questions/426390/how-can-i-rename-the-header-titles-of-the-glossary
\addto\captionsngerman{%
\renewcommand*{\glossaryname}{Begriffserklärungen}%
}


 
%%%%% %%%%% %%%%% %%%%% %%%%% \begin{document} %%%%% %%%%% %%%%% %%%%% %%%%%
\begin{document}

\maketitle

\section{Einleitung}
Im ersten Kapitel findet zu Beginn eine Erklärung der wichtigsten Begriffe und Abkürzungen statt. Dazu wird eine Themenabgrenzung durchgeführt sowie die Ausgangssituation beschrieben.

\printnoidxglossaries

\subsection{Themenabgrenzung}
Folgende Aspekte sind Bestandteil dieser Ausarbeitung:
\begin{itemize}
   \item Was ist die\gls{gls:DeNovo}?
   \item Was erhofft man sich von dieser Technologie?
   \item Welche Probleme liegen vor, die von der Seite der Informatik gelöst / verbessert werden können?
   \item Inwiefern spielen die Spektrums-Graphen dabei eine Rolle?
\end{itemize}

\subsection{Ausgangssituation}
Mit Hilfe der\gls{gls:DeNovo} ist grundsätzlich die Bestimmung von unbekannten Aminosäuresequenzen möglich. Das Verfahren arbeitet allerdings nicht in jeder Situation zuverlässig genug. Dadurch wird das Ermitteln von unbekannten Sequenzen erschwert. Auch bei bereits bekannten Sequenzen führt die nicht ausreichende Zuverlässigkeit dazu, dass bei Ergebnissen nicht sicher unterschieden werden kann, ob eine Änderung in der Aminosäuresequenz vorliegt oder ob fehlerhafte Daten bestimmt wurden.
Das Ziel ist mit Unterstützung von Software eine effizente Möglichkeit bereitzustellen, um die Zuverlässigkeit der\gls{gls:DeNovo} zu erhöhen. 



\section{De-Novo-Peptidsequenzierung und Spektrums-Graphen im Detail}
\subsection{Datengewinnung}
\subsection{Datenauswertung}
 

\section{Ergebnisse/Evaluierung}
\subsection{Probleme in der Praxis}
\subsection{Lösungsansätze}


\section{Zusammenfassung}
\subsection{Ungelöste Probleme}
\subsection{Kritische Betrachtung}


\section{Section}
\cite{OpenPNovo}
\cite{pNovoPlus}

\begingroup
\setlength{\emergencystretch}{.5em}
\printbibliography
\endgroup

\appendix
\section{Appendix Section}
Appendix



\end{document}
%%%%% %%%%% %%%%% %%%%% %%%%% \end{document} %%%%% %%%%% %%%%% %%%%% %%%%%

\documentclass[a4paper, 12pt]{article}
\usepackage[utf8]{inputenc} % UTF-8 Kodierung verwenden
\usepackage[backend=biber, sorting=none]{biblatex}
\usepackage[total={6.5in, 9in}]{geometry}
% \usepackage[onehalfspacing]{setspace} % 1.5 Spacing 
\usepackage[singlespacing]{setspace} % 1 Spacing 
\usepackage[T1]{fontenc}    % Fonts mit westeuropäischer Codierung verwenden
\usepackage[ngerman]{babel} % Neue deutsche Sprache
\usepackage{fancyhdr}       % Kopf- und Fusszeilen
\usepackage{tikz}           % Fuer das Erstellen von einfachen Grafiken
\usepackage{float}          % Fuer den Positionierungsbefehl '[H]'
\usepackage{fancyhdr}       % Angepasste Header und Footer
\usepackage{titling}        % Fuer Befehle wie \thetitle
% \usepackage{showframe}     % Boxen mit Rand visualisieren (nur für das Schreiben des Dokuments brauchbar!)
\usepackage{csquotes}
\usepackage{translator}
\usepackage[
nonumberlist, %keine Seitenzahlen anzeigen
%acronym,      %ein Abkürzungsverzeichnis erstellen
toc,          %Einträge im Inhaltsverzeichnis
section,      %im Inhaltsverzeichnis auf section-Ebene erscheinen
nopostdot     %Den Punkt am Ende jeder Beschreibung deaktivieren
]{glossaries}
\makenoidxglossaries
 
% \setlength{\abovecaptionskip}{1ex}
% \setlength{\belowcaptionskip}{1ex}
\setlength{\floatsep}{24pt}
\setlength{\textfloatsep}{24pt}
\setlength{\headheight}{15pt}

\setcounter{tocdepth}{1}

\title{De-Novo-Sequencing using Spectrum-Graphs, enabling Open Searches}
\author{Dominik Habermann}
\date{\today}

% Kopf- und Fussnoten anpassen
\pagestyle{fancy}
\fancyhf{}
\fancyhead[L]{\thetitle}
%\fancyhead[R]{\thetitle} 
\fancyfoot[C]{\thepage}



\addbibresource{P2_De-Novo-Sequencing using Spectrum-Graphs.bib}
\documentclass[a4paper, 12pt]{article}
\usepackage[utf8]{inputenc} % UTF-8 Kodierung verwenden
\usepackage[backend=biber, sorting=none]{biblatex}
\usepackage[total={6.5in, 9in}]{geometry}
% \usepackage[onehalfspacing]{setspace} % 1.5 Spacing 
\usepackage[singlespacing]{setspace} % 1 Spacing 
\usepackage[T1]{fontenc}    % Fonts mit westeuropäischer Codierung verwenden
\usepackage[ngerman]{babel} % Neue deutsche Sprache
\usepackage{fancyhdr}       % Kopf- und Fusszeilen
\usepackage{tikz}           % Fuer das Erstellen von einfachen Grafiken
\usepackage{float}          % Fuer den Positionierungsbefehl '[H]'
\usepackage{fancyhdr}       % Angepasste Header und Footer
\usepackage{titling}        % Fuer Befehle wie \thetitle
% \usepackage{showframe}     % Boxen mit Rand visualisieren (nur für das Schreiben des Dokuments brauchbar!)
\usepackage{csquotes}
\usepackage{translator}
\usepackage[
nonumberlist, %keine Seitenzahlen anzeigen
%acronym,      %ein Abkürzungsverzeichnis erstellen
toc,          %Einträge im Inhaltsverzeichnis
section,      %im Inhaltsverzeichnis auf section-Ebene erscheinen
nopostdot     %Den Punkt am Ende jeder Beschreibung deaktivieren
]{glossaries}
\makenoidxglossaries
 
% \setlength{\abovecaptionskip}{1ex}
% \setlength{\belowcaptionskip}{1ex}
\setlength{\floatsep}{24pt}
\setlength{\textfloatsep}{24pt}
\setlength{\headheight}{15pt}

\setcounter{tocdepth}{1}

\title{De-Novo-Sequencing using Spectrum-Graphs, enabling Open Searches}
\author{Dominik Habermann}
\date{\today}

% Kopf- und Fussnoten anpassen
\pagestyle{fancy}
\fancyhf{}
\fancyhead[L]{\thetitle}
%\fancyhead[R]{\thetitle} 
\fancyfoot[C]{\thepage}



\addbibresource{P2_De-Novo-Sequencing using Spectrum-Graphs.bib}
\input{P2_De-Novo-Sequencing using Spectrum-Graphs.gls}
\input{P2_De-Novo-Sequencing using Spectrum-Graphs.acr}

\newcommand{\gerquot}[1]{\glqq#1\grqq}
\newcommand{\dashAndSpace}{\textendash \space}

\renewcommand{\floatpagefraction}{0.8}
% Workaround um die Überschrift des Glossars anzupassen
% Siehe: https://tex.stackexchange.com/questions/426390/how-can-i-rename-the-header-titles-of-the-glossary
\addto\captionsngerman{%
\renewcommand*{\glossaryname}{Begriffserklärungen}%
}


 
%%%%% %%%%% %%%%% %%%%% %%%%% \begin{document} %%%%% %%%%% %%%%% %%%%% %%%%%
\begin{document}

\maketitle

\section{Einleitung}
Im ersten Kapitel findet zu Beginn eine Erklärung der wichtigsten Begriffe und Abkürzungen statt. Dazu wird eine Themenabgrenzung durchgeführt sowie die Ausgangssituation beschrieben.

\printnoidxglossaries

\subsection{Themenabgrenzung}
Folgende Aspekte sind Bestandteil dieser Ausarbeitung:
\begin{itemize}
   \item Was ist die\gls{gls:DeNovo}?
   \item Was erhofft man sich von dieser Technologie?
   \item Welche Probleme liegen vor, die von der Seite der Informatik gelöst / verbessert werden können?
   \item Inwiefern spielen die Spektrums-Graphen dabei eine Rolle?
\end{itemize}

\subsection{Ausgangssituation}
Mit Hilfe der\gls{gls:DeNovo} ist grundsätzlich die Bestimmung von unbekannten Aminosäuresequenzen möglich. Das Verfahren arbeitet allerdings nicht in jeder Situation zuverlässig genug. Dadurch wird das Ermitteln von unbekannten Sequenzen erschwert. Auch bei bereits bekannten Sequenzen führt die nicht ausreichende Zuverlässigkeit dazu, dass bei Ergebnissen nicht sicher unterschieden werden kann, ob eine Änderung in der Aminosäuresequenz vorliegt oder ob fehlerhafte Daten bestimmt wurden.
Das Ziel ist mit Unterstützung von Software eine effizente Möglichkeit bereitzustellen, um die Zuverlässigkeit der\gls{gls:DeNovo} zu erhöhen. 



\section{De-Novo-Peptidsequenzierung und Spektrums-Graphen im Detail}
\subsection{Datengewinnung}
\subsection{Datenauswertung}
 

\section{Ergebnisse/Evaluierung}
\subsection{Probleme in der Praxis}
\subsection{Lösungsansätze}


\section{Zusammenfassung}
\subsection{Ungelöste Probleme}
\subsection{Kritische Betrachtung}


\section{Section}
\cite{OpenPNovo}
\cite{pNovoPlus}

\begingroup
\setlength{\emergencystretch}{.5em}
\printbibliography
\endgroup

\appendix
\section{Appendix Section}
Appendix



\end{document}
%%%%% %%%%% %%%%% %%%%% %%%%% \end{document} %%%%% %%%%% %%%%% %%%%% %%%%%

\documentclass[a4paper, 12pt]{article}
\usepackage[utf8]{inputenc} % UTF-8 Kodierung verwenden
\usepackage[backend=biber, sorting=none]{biblatex}
\usepackage[total={6.5in, 9in}]{geometry}
% \usepackage[onehalfspacing]{setspace} % 1.5 Spacing 
\usepackage[singlespacing]{setspace} % 1 Spacing 
\usepackage[T1]{fontenc}    % Fonts mit westeuropäischer Codierung verwenden
\usepackage[ngerman]{babel} % Neue deutsche Sprache
\usepackage{fancyhdr}       % Kopf- und Fusszeilen
\usepackage{tikz}           % Fuer das Erstellen von einfachen Grafiken
\usepackage{float}          % Fuer den Positionierungsbefehl '[H]'
\usepackage{fancyhdr}       % Angepasste Header und Footer
\usepackage{titling}        % Fuer Befehle wie \thetitle
% \usepackage{showframe}     % Boxen mit Rand visualisieren (nur für das Schreiben des Dokuments brauchbar!)
\usepackage{csquotes}
\usepackage{translator}
\usepackage[
nonumberlist, %keine Seitenzahlen anzeigen
%acronym,      %ein Abkürzungsverzeichnis erstellen
toc,          %Einträge im Inhaltsverzeichnis
section,      %im Inhaltsverzeichnis auf section-Ebene erscheinen
nopostdot     %Den Punkt am Ende jeder Beschreibung deaktivieren
]{glossaries}
\makenoidxglossaries
 
% \setlength{\abovecaptionskip}{1ex}
% \setlength{\belowcaptionskip}{1ex}
\setlength{\floatsep}{24pt}
\setlength{\textfloatsep}{24pt}
\setlength{\headheight}{15pt}

\setcounter{tocdepth}{1}

\title{De-Novo-Sequencing using Spectrum-Graphs, enabling Open Searches}
\author{Dominik Habermann}
\date{\today}

% Kopf- und Fussnoten anpassen
\pagestyle{fancy}
\fancyhf{}
\fancyhead[L]{\thetitle}
%\fancyhead[R]{\thetitle} 
\fancyfoot[C]{\thepage}



\addbibresource{P2_De-Novo-Sequencing using Spectrum-Graphs.bib}
\input{P2_De-Novo-Sequencing using Spectrum-Graphs.gls}
\input{P2_De-Novo-Sequencing using Spectrum-Graphs.acr}

\newcommand{\gerquot}[1]{\glqq#1\grqq}
\newcommand{\dashAndSpace}{\textendash \space}

\renewcommand{\floatpagefraction}{0.8}
% Workaround um die Überschrift des Glossars anzupassen
% Siehe: https://tex.stackexchange.com/questions/426390/how-can-i-rename-the-header-titles-of-the-glossary
\addto\captionsngerman{%
\renewcommand*{\glossaryname}{Begriffserklärungen}%
}


 
%%%%% %%%%% %%%%% %%%%% %%%%% \begin{document} %%%%% %%%%% %%%%% %%%%% %%%%%
\begin{document}

\maketitle

\section{Einleitung}
Im ersten Kapitel findet zu Beginn eine Erklärung der wichtigsten Begriffe und Abkürzungen statt. Dazu wird eine Themenabgrenzung durchgeführt sowie die Ausgangssituation beschrieben.

\printnoidxglossaries

\subsection{Themenabgrenzung}
Folgende Aspekte sind Bestandteil dieser Ausarbeitung:
\begin{itemize}
   \item Was ist die\gls{gls:DeNovo}?
   \item Was erhofft man sich von dieser Technologie?
   \item Welche Probleme liegen vor, die von der Seite der Informatik gelöst / verbessert werden können?
   \item Inwiefern spielen die Spektrums-Graphen dabei eine Rolle?
\end{itemize}

\subsection{Ausgangssituation}
Mit Hilfe der\gls{gls:DeNovo} ist grundsätzlich die Bestimmung von unbekannten Aminosäuresequenzen möglich. Das Verfahren arbeitet allerdings nicht in jeder Situation zuverlässig genug. Dadurch wird das Ermitteln von unbekannten Sequenzen erschwert. Auch bei bereits bekannten Sequenzen führt die nicht ausreichende Zuverlässigkeit dazu, dass bei Ergebnissen nicht sicher unterschieden werden kann, ob eine Änderung in der Aminosäuresequenz vorliegt oder ob fehlerhafte Daten bestimmt wurden.
Das Ziel ist mit Unterstützung von Software eine effizente Möglichkeit bereitzustellen, um die Zuverlässigkeit der\gls{gls:DeNovo} zu erhöhen. 



\section{De-Novo-Peptidsequenzierung und Spektrums-Graphen im Detail}
\subsection{Datengewinnung}
\subsection{Datenauswertung}
 

\section{Ergebnisse/Evaluierung}
\subsection{Probleme in der Praxis}
\subsection{Lösungsansätze}


\section{Zusammenfassung}
\subsection{Ungelöste Probleme}
\subsection{Kritische Betrachtung}


\section{Section}
\cite{OpenPNovo}
\cite{pNovoPlus}

\begingroup
\setlength{\emergencystretch}{.5em}
\printbibliography
\endgroup

\appendix
\section{Appendix Section}
Appendix



\end{document}
%%%%% %%%%% %%%%% %%%%% %%%%% \end{document} %%%%% %%%%% %%%%% %%%%% %%%%%


\newcommand{\gerquot}[1]{\glqq#1\grqq}
\newcommand{\dashAndSpace}{\textendash \space}

\renewcommand{\floatpagefraction}{0.8}
% Workaround um die Überschrift des Glossars anzupassen
% Siehe: https://tex.stackexchange.com/questions/426390/how-can-i-rename-the-header-titles-of-the-glossary
\addto\captionsngerman{%
\renewcommand*{\glossaryname}{Begriffserklärungen}%
}


 
%%%%% %%%%% %%%%% %%%%% %%%%% \begin{document} %%%%% %%%%% %%%%% %%%%% %%%%%
\begin{document}

\maketitle

\section{Einleitung}
Im ersten Kapitel findet zu Beginn eine Erklärung der wichtigsten Begriffe und Abkürzungen statt. Dazu wird eine Themenabgrenzung durchgeführt sowie die Ausgangssituation beschrieben.

\printnoidxglossaries

\subsection{Themenabgrenzung}
Folgende Aspekte sind Bestandteil dieser Ausarbeitung:
\begin{itemize}
   \item Was ist die\gls{gls:DeNovo}?
   \item Was erhofft man sich von dieser Technologie?
   \item Welche Probleme liegen vor, die von der Seite der Informatik gelöst / verbessert werden können?
   \item Inwiefern spielen die Spektrums-Graphen dabei eine Rolle?
\end{itemize}

\subsection{Ausgangssituation}
Mit Hilfe der\gls{gls:DeNovo} ist grundsätzlich die Bestimmung von unbekannten Aminosäuresequenzen möglich. Das Verfahren arbeitet allerdings nicht in jeder Situation zuverlässig genug. Dadurch wird das Ermitteln von unbekannten Sequenzen erschwert. Auch bei bereits bekannten Sequenzen führt die nicht ausreichende Zuverlässigkeit dazu, dass bei Ergebnissen nicht sicher unterschieden werden kann, ob eine Änderung in der Aminosäuresequenz vorliegt oder ob fehlerhafte Daten bestimmt wurden.
Das Ziel ist mit Unterstützung von Software eine effizente Möglichkeit bereitzustellen, um die Zuverlässigkeit der\gls{gls:DeNovo} zu erhöhen. 



\section{De-Novo-Peptidsequenzierung und Spektrums-Graphen im Detail}
\subsection{Datengewinnung}
\subsection{Datenauswertung}
 

\section{Ergebnisse/Evaluierung}
\subsection{Probleme in der Praxis}
\subsection{Lösungsansätze}


\section{Zusammenfassung}
\subsection{Ungelöste Probleme}
\subsection{Kritische Betrachtung}


\section{Section}
\cite{OpenPNovo}
\cite{pNovoPlus}

\begingroup
\setlength{\emergencystretch}{.5em}
\printbibliography
\endgroup

\appendix
\section{Appendix Section}
Appendix



\end{document}
%%%%% %%%%% %%%%% %%%%% %%%%% \end{document} %%%%% %%%%% %%%%% %%%%% %%%%%


\newcommand{\gerquot}[1]{\glqq#1\grqq}
\newcommand{\dashAndSpace}{\textendash \space}

\renewcommand{\floatpagefraction}{0.8}
% Workaround um die Überschrift des Glossars anzupassen
% Siehe: https://tex.stackexchange.com/questions/426390/how-can-i-rename-the-header-titles-of-the-glossary
\addto\captionsngerman{%
\renewcommand*{\glossaryname}{Begriffserklärungen}%
}


 
%%%%% %%%%% %%%%% %%%%% %%%%% \begin{document} %%%%% %%%%% %%%%% %%%%% %%%%%
\begin{document}

\maketitle

\section{Einleitung}
Im ersten Kapitel findet zu Beginn eine Erklärung der wichtigsten Begriffe und Abkürzungen statt. Dazu wird eine Themenabgrenzung durchgeführt sowie die Ausgangssituation beschrieben.

\printnoidxglossaries

\subsection{Themenabgrenzung}
Folgende Aspekte sind Bestandteil dieser Ausarbeitung:
\begin{itemize}
   \item Was ist die\gls{gls:DeNovo}?
   \item Was erhofft man sich von dieser Technologie?
   \item Welche Probleme liegen vor, die von der Seite der Informatik gelöst / verbessert werden können?
   \item Inwiefern spielen die Spektrums-Graphen dabei eine Rolle?
\end{itemize}

\subsection{Ausgangssituation}
Mit Hilfe der\gls{gls:DeNovo} ist grundsätzlich die Bestimmung von unbekannten Aminosäuresequenzen möglich. Das Verfahren arbeitet allerdings nicht in jeder Situation zuverlässig genug. Dadurch wird das Ermitteln von unbekannten Sequenzen erschwert. Auch bei bereits bekannten Sequenzen führt die nicht ausreichende Zuverlässigkeit dazu, dass bei Ergebnissen nicht sicher unterschieden werden kann, ob eine Änderung in der Aminosäuresequenz vorliegt oder ob fehlerhafte Daten bestimmt wurden.
Das Ziel ist mit Unterstützung von Software eine effizente Möglichkeit bereitzustellen, um die Zuverlässigkeit der\gls{gls:DeNovo} zu erhöhen. 



\section{De-Novo-Peptidsequenzierung und Spektrums-Graphen im Detail}
\subsection{Datengewinnung}
\subsection{Datenauswertung}
 

\section{Ergebnisse/Evaluierung}
\subsection{Probleme in der Praxis}
\subsection{Lösungsansätze}


\section{Zusammenfassung}
\subsection{Ungelöste Probleme}
\subsection{Kritische Betrachtung}


\section{Section}
\cite{OpenPNovo}
\cite{pNovoPlus}

\begingroup
\setlength{\emergencystretch}{.5em}
\printbibliography
\endgroup

\appendix
\section{Appendix Section}
Appendix



\end{document}
%%%%% %%%%% %%%%% %%%%% %%%%% \end{document} %%%%% %%%%% %%%%% %%%%% %%%%%

\documentclass[a4paper, 12pt]{article}
\usepackage[utf8]{inputenc} % UTF-8 Kodierung verwenden
\usepackage[backend=biber, sorting=none]{biblatex}
\usepackage[total={6.5in, 9in}]{geometry}
% \usepackage[onehalfspacing]{setspace} % 1.5 Spacing 
\usepackage[singlespacing]{setspace} % 1 Spacing 
\usepackage[T1]{fontenc}    % Fonts mit westeuropäischer Codierung verwenden
\usepackage[ngerman]{babel} % Neue deutsche Sprache
\usepackage{fancyhdr}       % Kopf- und Fusszeilen
\usepackage{tikz}           % Fuer das Erstellen von einfachen Grafiken
\usepackage{float}          % Fuer den Positionierungsbefehl '[H]'
\usepackage{fancyhdr}       % Angepasste Header und Footer
\usepackage{titling}        % Fuer Befehle wie \thetitle
% \usepackage{showframe}     % Boxen mit Rand visualisieren (nur für das Schreiben des Dokuments brauchbar!)
\usepackage{csquotes}
\usepackage{translator}
\usepackage[
nonumberlist, %keine Seitenzahlen anzeigen
%acronym,      %ein Abkürzungsverzeichnis erstellen
toc,          %Einträge im Inhaltsverzeichnis
section,      %im Inhaltsverzeichnis auf section-Ebene erscheinen
nopostdot     %Den Punkt am Ende jeder Beschreibung deaktivieren
]{glossaries}
\makenoidxglossaries
 
% \setlength{\abovecaptionskip}{1ex}
% \setlength{\belowcaptionskip}{1ex}
\setlength{\floatsep}{24pt}
\setlength{\textfloatsep}{24pt}
\setlength{\headheight}{15pt}

\setcounter{tocdepth}{1}

\title{De-Novo-Sequencing using Spectrum-Graphs, enabling Open Searches}
\author{Dominik Habermann}
\date{\today}

% Kopf- und Fussnoten anpassen
\pagestyle{fancy}
\fancyhf{}
\fancyhead[L]{\thetitle}
%\fancyhead[R]{\thetitle} 
\fancyfoot[C]{\thepage}



\addbibresource{P2_De-Novo-Sequencing using Spectrum-Graphs.bib}
\documentclass[a4paper, 12pt]{article}
\usepackage[utf8]{inputenc} % UTF-8 Kodierung verwenden
\usepackage[backend=biber, sorting=none]{biblatex}
\usepackage[total={6.5in, 9in}]{geometry}
% \usepackage[onehalfspacing]{setspace} % 1.5 Spacing 
\usepackage[singlespacing]{setspace} % 1 Spacing 
\usepackage[T1]{fontenc}    % Fonts mit westeuropäischer Codierung verwenden
\usepackage[ngerman]{babel} % Neue deutsche Sprache
\usepackage{fancyhdr}       % Kopf- und Fusszeilen
\usepackage{tikz}           % Fuer das Erstellen von einfachen Grafiken
\usepackage{float}          % Fuer den Positionierungsbefehl '[H]'
\usepackage{fancyhdr}       % Angepasste Header und Footer
\usepackage{titling}        % Fuer Befehle wie \thetitle
% \usepackage{showframe}     % Boxen mit Rand visualisieren (nur für das Schreiben des Dokuments brauchbar!)
\usepackage{csquotes}
\usepackage{translator}
\usepackage[
nonumberlist, %keine Seitenzahlen anzeigen
%acronym,      %ein Abkürzungsverzeichnis erstellen
toc,          %Einträge im Inhaltsverzeichnis
section,      %im Inhaltsverzeichnis auf section-Ebene erscheinen
nopostdot     %Den Punkt am Ende jeder Beschreibung deaktivieren
]{glossaries}
\makenoidxglossaries
 
% \setlength{\abovecaptionskip}{1ex}
% \setlength{\belowcaptionskip}{1ex}
\setlength{\floatsep}{24pt}
\setlength{\textfloatsep}{24pt}
\setlength{\headheight}{15pt}

\setcounter{tocdepth}{1}

\title{De-Novo-Sequencing using Spectrum-Graphs, enabling Open Searches}
\author{Dominik Habermann}
\date{\today}

% Kopf- und Fussnoten anpassen
\pagestyle{fancy}
\fancyhf{}
\fancyhead[L]{\thetitle}
%\fancyhead[R]{\thetitle} 
\fancyfoot[C]{\thepage}



\addbibresource{P2_De-Novo-Sequencing using Spectrum-Graphs.bib}
\documentclass[a4paper, 12pt]{article}
\usepackage[utf8]{inputenc} % UTF-8 Kodierung verwenden
\usepackage[backend=biber, sorting=none]{biblatex}
\usepackage[total={6.5in, 9in}]{geometry}
% \usepackage[onehalfspacing]{setspace} % 1.5 Spacing 
\usepackage[singlespacing]{setspace} % 1 Spacing 
\usepackage[T1]{fontenc}    % Fonts mit westeuropäischer Codierung verwenden
\usepackage[ngerman]{babel} % Neue deutsche Sprache
\usepackage{fancyhdr}       % Kopf- und Fusszeilen
\usepackage{tikz}           % Fuer das Erstellen von einfachen Grafiken
\usepackage{float}          % Fuer den Positionierungsbefehl '[H]'
\usepackage{fancyhdr}       % Angepasste Header und Footer
\usepackage{titling}        % Fuer Befehle wie \thetitle
% \usepackage{showframe}     % Boxen mit Rand visualisieren (nur für das Schreiben des Dokuments brauchbar!)
\usepackage{csquotes}
\usepackage{translator}
\usepackage[
nonumberlist, %keine Seitenzahlen anzeigen
%acronym,      %ein Abkürzungsverzeichnis erstellen
toc,          %Einträge im Inhaltsverzeichnis
section,      %im Inhaltsverzeichnis auf section-Ebene erscheinen
nopostdot     %Den Punkt am Ende jeder Beschreibung deaktivieren
]{glossaries}
\makenoidxglossaries
 
% \setlength{\abovecaptionskip}{1ex}
% \setlength{\belowcaptionskip}{1ex}
\setlength{\floatsep}{24pt}
\setlength{\textfloatsep}{24pt}
\setlength{\headheight}{15pt}

\setcounter{tocdepth}{1}

\title{De-Novo-Sequencing using Spectrum-Graphs, enabling Open Searches}
\author{Dominik Habermann}
\date{\today}

% Kopf- und Fussnoten anpassen
\pagestyle{fancy}
\fancyhf{}
\fancyhead[L]{\thetitle}
%\fancyhead[R]{\thetitle} 
\fancyfoot[C]{\thepage}



\addbibresource{P2_De-Novo-Sequencing using Spectrum-Graphs.bib}
\input{P2_De-Novo-Sequencing using Spectrum-Graphs.gls}
\input{P2_De-Novo-Sequencing using Spectrum-Graphs.acr}

\newcommand{\gerquot}[1]{\glqq#1\grqq}
\newcommand{\dashAndSpace}{\textendash \space}

\renewcommand{\floatpagefraction}{0.8}
% Workaround um die Überschrift des Glossars anzupassen
% Siehe: https://tex.stackexchange.com/questions/426390/how-can-i-rename-the-header-titles-of-the-glossary
\addto\captionsngerman{%
\renewcommand*{\glossaryname}{Begriffserklärungen}%
}


 
%%%%% %%%%% %%%%% %%%%% %%%%% \begin{document} %%%%% %%%%% %%%%% %%%%% %%%%%
\begin{document}

\maketitle

\section{Einleitung}
Im ersten Kapitel findet zu Beginn eine Erklärung der wichtigsten Begriffe und Abkürzungen statt. Dazu wird eine Themenabgrenzung durchgeführt sowie die Ausgangssituation beschrieben.

\printnoidxglossaries

\subsection{Themenabgrenzung}
Folgende Aspekte sind Bestandteil dieser Ausarbeitung:
\begin{itemize}
   \item Was ist die\gls{gls:DeNovo}?
   \item Was erhofft man sich von dieser Technologie?
   \item Welche Probleme liegen vor, die von der Seite der Informatik gelöst / verbessert werden können?
   \item Inwiefern spielen die Spektrums-Graphen dabei eine Rolle?
\end{itemize}

\subsection{Ausgangssituation}
Mit Hilfe der\gls{gls:DeNovo} ist grundsätzlich die Bestimmung von unbekannten Aminosäuresequenzen möglich. Das Verfahren arbeitet allerdings nicht in jeder Situation zuverlässig genug. Dadurch wird das Ermitteln von unbekannten Sequenzen erschwert. Auch bei bereits bekannten Sequenzen führt die nicht ausreichende Zuverlässigkeit dazu, dass bei Ergebnissen nicht sicher unterschieden werden kann, ob eine Änderung in der Aminosäuresequenz vorliegt oder ob fehlerhafte Daten bestimmt wurden.
Das Ziel ist mit Unterstützung von Software eine effizente Möglichkeit bereitzustellen, um die Zuverlässigkeit der\gls{gls:DeNovo} zu erhöhen. 



\section{De-Novo-Peptidsequenzierung und Spektrums-Graphen im Detail}
\subsection{Datengewinnung}
\subsection{Datenauswertung}
 

\section{Ergebnisse/Evaluierung}
\subsection{Probleme in der Praxis}
\subsection{Lösungsansätze}


\section{Zusammenfassung}
\subsection{Ungelöste Probleme}
\subsection{Kritische Betrachtung}


\section{Section}
\cite{OpenPNovo}
\cite{pNovoPlus}

\begingroup
\setlength{\emergencystretch}{.5em}
\printbibliography
\endgroup

\appendix
\section{Appendix Section}
Appendix



\end{document}
%%%%% %%%%% %%%%% %%%%% %%%%% \end{document} %%%%% %%%%% %%%%% %%%%% %%%%%

\documentclass[a4paper, 12pt]{article}
\usepackage[utf8]{inputenc} % UTF-8 Kodierung verwenden
\usepackage[backend=biber, sorting=none]{biblatex}
\usepackage[total={6.5in, 9in}]{geometry}
% \usepackage[onehalfspacing]{setspace} % 1.5 Spacing 
\usepackage[singlespacing]{setspace} % 1 Spacing 
\usepackage[T1]{fontenc}    % Fonts mit westeuropäischer Codierung verwenden
\usepackage[ngerman]{babel} % Neue deutsche Sprache
\usepackage{fancyhdr}       % Kopf- und Fusszeilen
\usepackage{tikz}           % Fuer das Erstellen von einfachen Grafiken
\usepackage{float}          % Fuer den Positionierungsbefehl '[H]'
\usepackage{fancyhdr}       % Angepasste Header und Footer
\usepackage{titling}        % Fuer Befehle wie \thetitle
% \usepackage{showframe}     % Boxen mit Rand visualisieren (nur für das Schreiben des Dokuments brauchbar!)
\usepackage{csquotes}
\usepackage{translator}
\usepackage[
nonumberlist, %keine Seitenzahlen anzeigen
%acronym,      %ein Abkürzungsverzeichnis erstellen
toc,          %Einträge im Inhaltsverzeichnis
section,      %im Inhaltsverzeichnis auf section-Ebene erscheinen
nopostdot     %Den Punkt am Ende jeder Beschreibung deaktivieren
]{glossaries}
\makenoidxglossaries
 
% \setlength{\abovecaptionskip}{1ex}
% \setlength{\belowcaptionskip}{1ex}
\setlength{\floatsep}{24pt}
\setlength{\textfloatsep}{24pt}
\setlength{\headheight}{15pt}

\setcounter{tocdepth}{1}

\title{De-Novo-Sequencing using Spectrum-Graphs, enabling Open Searches}
\author{Dominik Habermann}
\date{\today}

% Kopf- und Fussnoten anpassen
\pagestyle{fancy}
\fancyhf{}
\fancyhead[L]{\thetitle}
%\fancyhead[R]{\thetitle} 
\fancyfoot[C]{\thepage}



\addbibresource{P2_De-Novo-Sequencing using Spectrum-Graphs.bib}
\input{P2_De-Novo-Sequencing using Spectrum-Graphs.gls}
\input{P2_De-Novo-Sequencing using Spectrum-Graphs.acr}

\newcommand{\gerquot}[1]{\glqq#1\grqq}
\newcommand{\dashAndSpace}{\textendash \space}

\renewcommand{\floatpagefraction}{0.8}
% Workaround um die Überschrift des Glossars anzupassen
% Siehe: https://tex.stackexchange.com/questions/426390/how-can-i-rename-the-header-titles-of-the-glossary
\addto\captionsngerman{%
\renewcommand*{\glossaryname}{Begriffserklärungen}%
}


 
%%%%% %%%%% %%%%% %%%%% %%%%% \begin{document} %%%%% %%%%% %%%%% %%%%% %%%%%
\begin{document}

\maketitle

\section{Einleitung}
Im ersten Kapitel findet zu Beginn eine Erklärung der wichtigsten Begriffe und Abkürzungen statt. Dazu wird eine Themenabgrenzung durchgeführt sowie die Ausgangssituation beschrieben.

\printnoidxglossaries

\subsection{Themenabgrenzung}
Folgende Aspekte sind Bestandteil dieser Ausarbeitung:
\begin{itemize}
   \item Was ist die\gls{gls:DeNovo}?
   \item Was erhofft man sich von dieser Technologie?
   \item Welche Probleme liegen vor, die von der Seite der Informatik gelöst / verbessert werden können?
   \item Inwiefern spielen die Spektrums-Graphen dabei eine Rolle?
\end{itemize}

\subsection{Ausgangssituation}
Mit Hilfe der\gls{gls:DeNovo} ist grundsätzlich die Bestimmung von unbekannten Aminosäuresequenzen möglich. Das Verfahren arbeitet allerdings nicht in jeder Situation zuverlässig genug. Dadurch wird das Ermitteln von unbekannten Sequenzen erschwert. Auch bei bereits bekannten Sequenzen führt die nicht ausreichende Zuverlässigkeit dazu, dass bei Ergebnissen nicht sicher unterschieden werden kann, ob eine Änderung in der Aminosäuresequenz vorliegt oder ob fehlerhafte Daten bestimmt wurden.
Das Ziel ist mit Unterstützung von Software eine effizente Möglichkeit bereitzustellen, um die Zuverlässigkeit der\gls{gls:DeNovo} zu erhöhen. 



\section{De-Novo-Peptidsequenzierung und Spektrums-Graphen im Detail}
\subsection{Datengewinnung}
\subsection{Datenauswertung}
 

\section{Ergebnisse/Evaluierung}
\subsection{Probleme in der Praxis}
\subsection{Lösungsansätze}


\section{Zusammenfassung}
\subsection{Ungelöste Probleme}
\subsection{Kritische Betrachtung}


\section{Section}
\cite{OpenPNovo}
\cite{pNovoPlus}

\begingroup
\setlength{\emergencystretch}{.5em}
\printbibliography
\endgroup

\appendix
\section{Appendix Section}
Appendix



\end{document}
%%%%% %%%%% %%%%% %%%%% %%%%% \end{document} %%%%% %%%%% %%%%% %%%%% %%%%%


\newcommand{\gerquot}[1]{\glqq#1\grqq}
\newcommand{\dashAndSpace}{\textendash \space}

\renewcommand{\floatpagefraction}{0.8}
% Workaround um die Überschrift des Glossars anzupassen
% Siehe: https://tex.stackexchange.com/questions/426390/how-can-i-rename-the-header-titles-of-the-glossary
\addto\captionsngerman{%
\renewcommand*{\glossaryname}{Begriffserklärungen}%
}


 
%%%%% %%%%% %%%%% %%%%% %%%%% \begin{document} %%%%% %%%%% %%%%% %%%%% %%%%%
\begin{document}

\maketitle

\section{Einleitung}
Im ersten Kapitel findet zu Beginn eine Erklärung der wichtigsten Begriffe und Abkürzungen statt. Dazu wird eine Themenabgrenzung durchgeführt sowie die Ausgangssituation beschrieben.

\printnoidxglossaries

\subsection{Themenabgrenzung}
Folgende Aspekte sind Bestandteil dieser Ausarbeitung:
\begin{itemize}
   \item Was ist die\gls{gls:DeNovo}?
   \item Was erhofft man sich von dieser Technologie?
   \item Welche Probleme liegen vor, die von der Seite der Informatik gelöst / verbessert werden können?
   \item Inwiefern spielen die Spektrums-Graphen dabei eine Rolle?
\end{itemize}

\subsection{Ausgangssituation}
Mit Hilfe der\gls{gls:DeNovo} ist grundsätzlich die Bestimmung von unbekannten Aminosäuresequenzen möglich. Das Verfahren arbeitet allerdings nicht in jeder Situation zuverlässig genug. Dadurch wird das Ermitteln von unbekannten Sequenzen erschwert. Auch bei bereits bekannten Sequenzen führt die nicht ausreichende Zuverlässigkeit dazu, dass bei Ergebnissen nicht sicher unterschieden werden kann, ob eine Änderung in der Aminosäuresequenz vorliegt oder ob fehlerhafte Daten bestimmt wurden.
Das Ziel ist mit Unterstützung von Software eine effizente Möglichkeit bereitzustellen, um die Zuverlässigkeit der\gls{gls:DeNovo} zu erhöhen. 



\section{De-Novo-Peptidsequenzierung und Spektrums-Graphen im Detail}
\subsection{Datengewinnung}
\subsection{Datenauswertung}
 

\section{Ergebnisse/Evaluierung}
\subsection{Probleme in der Praxis}
\subsection{Lösungsansätze}


\section{Zusammenfassung}
\subsection{Ungelöste Probleme}
\subsection{Kritische Betrachtung}


\section{Section}
\cite{OpenPNovo}
\cite{pNovoPlus}

\begingroup
\setlength{\emergencystretch}{.5em}
\printbibliography
\endgroup

\appendix
\section{Appendix Section}
Appendix



\end{document}
%%%%% %%%%% %%%%% %%%%% %%%%% \end{document} %%%%% %%%%% %%%%% %%%%% %%%%%

\documentclass[a4paper, 12pt]{article}
\usepackage[utf8]{inputenc} % UTF-8 Kodierung verwenden
\usepackage[backend=biber, sorting=none]{biblatex}
\usepackage[total={6.5in, 9in}]{geometry}
% \usepackage[onehalfspacing]{setspace} % 1.5 Spacing 
\usepackage[singlespacing]{setspace} % 1 Spacing 
\usepackage[T1]{fontenc}    % Fonts mit westeuropäischer Codierung verwenden
\usepackage[ngerman]{babel} % Neue deutsche Sprache
\usepackage{fancyhdr}       % Kopf- und Fusszeilen
\usepackage{tikz}           % Fuer das Erstellen von einfachen Grafiken
\usepackage{float}          % Fuer den Positionierungsbefehl '[H]'
\usepackage{fancyhdr}       % Angepasste Header und Footer
\usepackage{titling}        % Fuer Befehle wie \thetitle
% \usepackage{showframe}     % Boxen mit Rand visualisieren (nur für das Schreiben des Dokuments brauchbar!)
\usepackage{csquotes}
\usepackage{translator}
\usepackage[
nonumberlist, %keine Seitenzahlen anzeigen
%acronym,      %ein Abkürzungsverzeichnis erstellen
toc,          %Einträge im Inhaltsverzeichnis
section,      %im Inhaltsverzeichnis auf section-Ebene erscheinen
nopostdot     %Den Punkt am Ende jeder Beschreibung deaktivieren
]{glossaries}
\makenoidxglossaries
 
% \setlength{\abovecaptionskip}{1ex}
% \setlength{\belowcaptionskip}{1ex}
\setlength{\floatsep}{24pt}
\setlength{\textfloatsep}{24pt}
\setlength{\headheight}{15pt}

\setcounter{tocdepth}{1}

\title{De-Novo-Sequencing using Spectrum-Graphs, enabling Open Searches}
\author{Dominik Habermann}
\date{\today}

% Kopf- und Fussnoten anpassen
\pagestyle{fancy}
\fancyhf{}
\fancyhead[L]{\thetitle}
%\fancyhead[R]{\thetitle} 
\fancyfoot[C]{\thepage}



\addbibresource{P2_De-Novo-Sequencing using Spectrum-Graphs.bib}
\documentclass[a4paper, 12pt]{article}
\usepackage[utf8]{inputenc} % UTF-8 Kodierung verwenden
\usepackage[backend=biber, sorting=none]{biblatex}
\usepackage[total={6.5in, 9in}]{geometry}
% \usepackage[onehalfspacing]{setspace} % 1.5 Spacing 
\usepackage[singlespacing]{setspace} % 1 Spacing 
\usepackage[T1]{fontenc}    % Fonts mit westeuropäischer Codierung verwenden
\usepackage[ngerman]{babel} % Neue deutsche Sprache
\usepackage{fancyhdr}       % Kopf- und Fusszeilen
\usepackage{tikz}           % Fuer das Erstellen von einfachen Grafiken
\usepackage{float}          % Fuer den Positionierungsbefehl '[H]'
\usepackage{fancyhdr}       % Angepasste Header und Footer
\usepackage{titling}        % Fuer Befehle wie \thetitle
% \usepackage{showframe}     % Boxen mit Rand visualisieren (nur für das Schreiben des Dokuments brauchbar!)
\usepackage{csquotes}
\usepackage{translator}
\usepackage[
nonumberlist, %keine Seitenzahlen anzeigen
%acronym,      %ein Abkürzungsverzeichnis erstellen
toc,          %Einträge im Inhaltsverzeichnis
section,      %im Inhaltsverzeichnis auf section-Ebene erscheinen
nopostdot     %Den Punkt am Ende jeder Beschreibung deaktivieren
]{glossaries}
\makenoidxglossaries
 
% \setlength{\abovecaptionskip}{1ex}
% \setlength{\belowcaptionskip}{1ex}
\setlength{\floatsep}{24pt}
\setlength{\textfloatsep}{24pt}
\setlength{\headheight}{15pt}

\setcounter{tocdepth}{1}

\title{De-Novo-Sequencing using Spectrum-Graphs, enabling Open Searches}
\author{Dominik Habermann}
\date{\today}

% Kopf- und Fussnoten anpassen
\pagestyle{fancy}
\fancyhf{}
\fancyhead[L]{\thetitle}
%\fancyhead[R]{\thetitle} 
\fancyfoot[C]{\thepage}



\addbibresource{P2_De-Novo-Sequencing using Spectrum-Graphs.bib}
\input{P2_De-Novo-Sequencing using Spectrum-Graphs.gls}
\input{P2_De-Novo-Sequencing using Spectrum-Graphs.acr}

\newcommand{\gerquot}[1]{\glqq#1\grqq}
\newcommand{\dashAndSpace}{\textendash \space}

\renewcommand{\floatpagefraction}{0.8}
% Workaround um die Überschrift des Glossars anzupassen
% Siehe: https://tex.stackexchange.com/questions/426390/how-can-i-rename-the-header-titles-of-the-glossary
\addto\captionsngerman{%
\renewcommand*{\glossaryname}{Begriffserklärungen}%
}


 
%%%%% %%%%% %%%%% %%%%% %%%%% \begin{document} %%%%% %%%%% %%%%% %%%%% %%%%%
\begin{document}

\maketitle

\section{Einleitung}
Im ersten Kapitel findet zu Beginn eine Erklärung der wichtigsten Begriffe und Abkürzungen statt. Dazu wird eine Themenabgrenzung durchgeführt sowie die Ausgangssituation beschrieben.

\printnoidxglossaries

\subsection{Themenabgrenzung}
Folgende Aspekte sind Bestandteil dieser Ausarbeitung:
\begin{itemize}
   \item Was ist die\gls{gls:DeNovo}?
   \item Was erhofft man sich von dieser Technologie?
   \item Welche Probleme liegen vor, die von der Seite der Informatik gelöst / verbessert werden können?
   \item Inwiefern spielen die Spektrums-Graphen dabei eine Rolle?
\end{itemize}

\subsection{Ausgangssituation}
Mit Hilfe der\gls{gls:DeNovo} ist grundsätzlich die Bestimmung von unbekannten Aminosäuresequenzen möglich. Das Verfahren arbeitet allerdings nicht in jeder Situation zuverlässig genug. Dadurch wird das Ermitteln von unbekannten Sequenzen erschwert. Auch bei bereits bekannten Sequenzen führt die nicht ausreichende Zuverlässigkeit dazu, dass bei Ergebnissen nicht sicher unterschieden werden kann, ob eine Änderung in der Aminosäuresequenz vorliegt oder ob fehlerhafte Daten bestimmt wurden.
Das Ziel ist mit Unterstützung von Software eine effizente Möglichkeit bereitzustellen, um die Zuverlässigkeit der\gls{gls:DeNovo} zu erhöhen. 



\section{De-Novo-Peptidsequenzierung und Spektrums-Graphen im Detail}
\subsection{Datengewinnung}
\subsection{Datenauswertung}
 

\section{Ergebnisse/Evaluierung}
\subsection{Probleme in der Praxis}
\subsection{Lösungsansätze}


\section{Zusammenfassung}
\subsection{Ungelöste Probleme}
\subsection{Kritische Betrachtung}


\section{Section}
\cite{OpenPNovo}
\cite{pNovoPlus}

\begingroup
\setlength{\emergencystretch}{.5em}
\printbibliography
\endgroup

\appendix
\section{Appendix Section}
Appendix



\end{document}
%%%%% %%%%% %%%%% %%%%% %%%%% \end{document} %%%%% %%%%% %%%%% %%%%% %%%%%

\documentclass[a4paper, 12pt]{article}
\usepackage[utf8]{inputenc} % UTF-8 Kodierung verwenden
\usepackage[backend=biber, sorting=none]{biblatex}
\usepackage[total={6.5in, 9in}]{geometry}
% \usepackage[onehalfspacing]{setspace} % 1.5 Spacing 
\usepackage[singlespacing]{setspace} % 1 Spacing 
\usepackage[T1]{fontenc}    % Fonts mit westeuropäischer Codierung verwenden
\usepackage[ngerman]{babel} % Neue deutsche Sprache
\usepackage{fancyhdr}       % Kopf- und Fusszeilen
\usepackage{tikz}           % Fuer das Erstellen von einfachen Grafiken
\usepackage{float}          % Fuer den Positionierungsbefehl '[H]'
\usepackage{fancyhdr}       % Angepasste Header und Footer
\usepackage{titling}        % Fuer Befehle wie \thetitle
% \usepackage{showframe}     % Boxen mit Rand visualisieren (nur für das Schreiben des Dokuments brauchbar!)
\usepackage{csquotes}
\usepackage{translator}
\usepackage[
nonumberlist, %keine Seitenzahlen anzeigen
%acronym,      %ein Abkürzungsverzeichnis erstellen
toc,          %Einträge im Inhaltsverzeichnis
section,      %im Inhaltsverzeichnis auf section-Ebene erscheinen
nopostdot     %Den Punkt am Ende jeder Beschreibung deaktivieren
]{glossaries}
\makenoidxglossaries
 
% \setlength{\abovecaptionskip}{1ex}
% \setlength{\belowcaptionskip}{1ex}
\setlength{\floatsep}{24pt}
\setlength{\textfloatsep}{24pt}
\setlength{\headheight}{15pt}

\setcounter{tocdepth}{1}

\title{De-Novo-Sequencing using Spectrum-Graphs, enabling Open Searches}
\author{Dominik Habermann}
\date{\today}

% Kopf- und Fussnoten anpassen
\pagestyle{fancy}
\fancyhf{}
\fancyhead[L]{\thetitle}
%\fancyhead[R]{\thetitle} 
\fancyfoot[C]{\thepage}



\addbibresource{P2_De-Novo-Sequencing using Spectrum-Graphs.bib}
\input{P2_De-Novo-Sequencing using Spectrum-Graphs.gls}
\input{P2_De-Novo-Sequencing using Spectrum-Graphs.acr}

\newcommand{\gerquot}[1]{\glqq#1\grqq}
\newcommand{\dashAndSpace}{\textendash \space}

\renewcommand{\floatpagefraction}{0.8}
% Workaround um die Überschrift des Glossars anzupassen
% Siehe: https://tex.stackexchange.com/questions/426390/how-can-i-rename-the-header-titles-of-the-glossary
\addto\captionsngerman{%
\renewcommand*{\glossaryname}{Begriffserklärungen}%
}


 
%%%%% %%%%% %%%%% %%%%% %%%%% \begin{document} %%%%% %%%%% %%%%% %%%%% %%%%%
\begin{document}

\maketitle

\section{Einleitung}
Im ersten Kapitel findet zu Beginn eine Erklärung der wichtigsten Begriffe und Abkürzungen statt. Dazu wird eine Themenabgrenzung durchgeführt sowie die Ausgangssituation beschrieben.

\printnoidxglossaries

\subsection{Themenabgrenzung}
Folgende Aspekte sind Bestandteil dieser Ausarbeitung:
\begin{itemize}
   \item Was ist die\gls{gls:DeNovo}?
   \item Was erhofft man sich von dieser Technologie?
   \item Welche Probleme liegen vor, die von der Seite der Informatik gelöst / verbessert werden können?
   \item Inwiefern spielen die Spektrums-Graphen dabei eine Rolle?
\end{itemize}

\subsection{Ausgangssituation}
Mit Hilfe der\gls{gls:DeNovo} ist grundsätzlich die Bestimmung von unbekannten Aminosäuresequenzen möglich. Das Verfahren arbeitet allerdings nicht in jeder Situation zuverlässig genug. Dadurch wird das Ermitteln von unbekannten Sequenzen erschwert. Auch bei bereits bekannten Sequenzen führt die nicht ausreichende Zuverlässigkeit dazu, dass bei Ergebnissen nicht sicher unterschieden werden kann, ob eine Änderung in der Aminosäuresequenz vorliegt oder ob fehlerhafte Daten bestimmt wurden.
Das Ziel ist mit Unterstützung von Software eine effizente Möglichkeit bereitzustellen, um die Zuverlässigkeit der\gls{gls:DeNovo} zu erhöhen. 



\section{De-Novo-Peptidsequenzierung und Spektrums-Graphen im Detail}
\subsection{Datengewinnung}
\subsection{Datenauswertung}
 

\section{Ergebnisse/Evaluierung}
\subsection{Probleme in der Praxis}
\subsection{Lösungsansätze}


\section{Zusammenfassung}
\subsection{Ungelöste Probleme}
\subsection{Kritische Betrachtung}


\section{Section}
\cite{OpenPNovo}
\cite{pNovoPlus}

\begingroup
\setlength{\emergencystretch}{.5em}
\printbibliography
\endgroup

\appendix
\section{Appendix Section}
Appendix



\end{document}
%%%%% %%%%% %%%%% %%%%% %%%%% \end{document} %%%%% %%%%% %%%%% %%%%% %%%%%


\newcommand{\gerquot}[1]{\glqq#1\grqq}
\newcommand{\dashAndSpace}{\textendash \space}

\renewcommand{\floatpagefraction}{0.8}
% Workaround um die Überschrift des Glossars anzupassen
% Siehe: https://tex.stackexchange.com/questions/426390/how-can-i-rename-the-header-titles-of-the-glossary
\addto\captionsngerman{%
\renewcommand*{\glossaryname}{Begriffserklärungen}%
}


 
%%%%% %%%%% %%%%% %%%%% %%%%% \begin{document} %%%%% %%%%% %%%%% %%%%% %%%%%
\begin{document}

\maketitle

\section{Einleitung}
Im ersten Kapitel findet zu Beginn eine Erklärung der wichtigsten Begriffe und Abkürzungen statt. Dazu wird eine Themenabgrenzung durchgeführt sowie die Ausgangssituation beschrieben.

\printnoidxglossaries

\subsection{Themenabgrenzung}
Folgende Aspekte sind Bestandteil dieser Ausarbeitung:
\begin{itemize}
   \item Was ist die\gls{gls:DeNovo}?
   \item Was erhofft man sich von dieser Technologie?
   \item Welche Probleme liegen vor, die von der Seite der Informatik gelöst / verbessert werden können?
   \item Inwiefern spielen die Spektrums-Graphen dabei eine Rolle?
\end{itemize}

\subsection{Ausgangssituation}
Mit Hilfe der\gls{gls:DeNovo} ist grundsätzlich die Bestimmung von unbekannten Aminosäuresequenzen möglich. Das Verfahren arbeitet allerdings nicht in jeder Situation zuverlässig genug. Dadurch wird das Ermitteln von unbekannten Sequenzen erschwert. Auch bei bereits bekannten Sequenzen führt die nicht ausreichende Zuverlässigkeit dazu, dass bei Ergebnissen nicht sicher unterschieden werden kann, ob eine Änderung in der Aminosäuresequenz vorliegt oder ob fehlerhafte Daten bestimmt wurden.
Das Ziel ist mit Unterstützung von Software eine effizente Möglichkeit bereitzustellen, um die Zuverlässigkeit der\gls{gls:DeNovo} zu erhöhen. 



\section{De-Novo-Peptidsequenzierung und Spektrums-Graphen im Detail}
\subsection{Datengewinnung}
\subsection{Datenauswertung}
 

\section{Ergebnisse/Evaluierung}
\subsection{Probleme in der Praxis}
\subsection{Lösungsansätze}


\section{Zusammenfassung}
\subsection{Ungelöste Probleme}
\subsection{Kritische Betrachtung}


\section{Section}
\cite{OpenPNovo}
\cite{pNovoPlus}

\begingroup
\setlength{\emergencystretch}{.5em}
\printbibliography
\endgroup

\appendix
\section{Appendix Section}
Appendix



\end{document}
%%%%% %%%%% %%%%% %%%%% %%%%% \end{document} %%%%% %%%%% %%%%% %%%%% %%%%%


\newcommand{\gerquot}[1]{\glqq#1\grqq}
\newcommand{\dashAndSpace}{\textendash \space}

\renewcommand{\floatpagefraction}{0.8}
% Workaround um die Überschrift des Glossars anzupassen
% Siehe: https://tex.stackexchange.com/questions/426390/how-can-i-rename-the-header-titles-of-the-glossary
\addto\captionsngerman{%
\renewcommand*{\glossaryname}{Begriffserklärungen}%
}


 
%%%%% %%%%% %%%%% %%%%% %%%%% \begin{document} %%%%% %%%%% %%%%% %%%%% %%%%%
\begin{document}

\maketitle

\section{Einleitung}
Im ersten Kapitel findet zu Beginn eine Erklärung der wichtigsten Begriffe und Abkürzungen statt. Dazu wird eine Themenabgrenzung durchgeführt sowie die Ausgangssituation beschrieben.

\printnoidxglossaries

\subsection{Themenabgrenzung}
Folgende Aspekte sind Bestandteil dieser Ausarbeitung:
\begin{itemize}
   \item Was ist die\gls{gls:DeNovo}?
   \item Was erhofft man sich von dieser Technologie?
   \item Welche Probleme liegen vor, die von der Seite der Informatik gelöst / verbessert werden können?
   \item Inwiefern spielen die Spektrums-Graphen dabei eine Rolle?
\end{itemize}

\subsection{Ausgangssituation}
Mit Hilfe der\gls{gls:DeNovo} ist grundsätzlich die Bestimmung von unbekannten Aminosäuresequenzen möglich. Das Verfahren arbeitet allerdings nicht in jeder Situation zuverlässig genug. Dadurch wird das Ermitteln von unbekannten Sequenzen erschwert. Auch bei bereits bekannten Sequenzen führt die nicht ausreichende Zuverlässigkeit dazu, dass bei Ergebnissen nicht sicher unterschieden werden kann, ob eine Änderung in der Aminosäuresequenz vorliegt oder ob fehlerhafte Daten bestimmt wurden.
Das Ziel ist mit Unterstützung von Software eine effizente Möglichkeit bereitzustellen, um die Zuverlässigkeit der\gls{gls:DeNovo} zu erhöhen. 



\section{De-Novo-Peptidsequenzierung und Spektrums-Graphen im Detail}
\subsection{Datengewinnung}
\subsection{Datenauswertung}
 

\section{Ergebnisse/Evaluierung}
\subsection{Probleme in der Praxis}
\subsection{Lösungsansätze}


\section{Zusammenfassung}
\subsection{Ungelöste Probleme}
\subsection{Kritische Betrachtung}


\section{Section}
\cite{OpenPNovo}
\cite{pNovoPlus}

\begingroup
\setlength{\emergencystretch}{.5em}
\printbibliography
\endgroup

\appendix
\section{Appendix Section}
Appendix



\end{document}
%%%%% %%%%% %%%%% %%%%% %%%%% \end{document} %%%%% %%%%% %%%%% %%%%% %%%%%


\newcommand{\gerquot}[1]{\glqq#1\grqq}
\newcommand{\dashAndSpace}{\textendash \space}

\renewcommand{\floatpagefraction}{0.8}
% Workaround um die Überschrift des Glossars anzupassen
% Siehe: https://tex.stackexchange.com/questions/426390/how-can-i-rename-the-header-titles-of-the-glossary
\addto\captionsngerman{%
\renewcommand*{\glossaryname}{Begriffserklärungen}%
}


 
%%%%% %%%%% %%%%% %%%%% %%%%% \begin{document} %%%%% %%%%% %%%%% %%%%% %%%%%
\begin{document}

\maketitle

\section{Einleitung}
Im ersten Kapitel findet zu Beginn eine Erklärung der wichtigsten Begriffe und Abkürzungen statt. Dazu wird eine Themenabgrenzung durchgeführt sowie die Ausgangssituation beschrieben.

\printnoidxglossaries

\subsection{Themenabgrenzung}
Folgende Aspekte sind Bestandteil dieser Ausarbeitung:
\begin{itemize}
   \item Was ist die\gls{gls:DeNovo}?
   \item Was erhofft man sich von dieser Technologie?
   \item Welche Probleme liegen vor, die von der Seite der Informatik gelöst / verbessert werden können?
   \item Inwiefern spielen die Spektrums-Graphen dabei eine Rolle?
\end{itemize}

\subsection{Ausgangssituation}
Mit Hilfe der\gls{gls:DeNovo} ist grundsätzlich die Bestimmung von unbekannten Aminosäuresequenzen möglich. Das Verfahren arbeitet allerdings nicht in jeder Situation zuverlässig genug. Dadurch wird das Ermitteln von unbekannten Sequenzen erschwert. Auch bei bereits bekannten Sequenzen führt die nicht ausreichende Zuverlässigkeit dazu, dass bei Ergebnissen nicht sicher unterschieden werden kann, ob eine Änderung in der Aminosäuresequenz vorliegt oder ob fehlerhafte Daten bestimmt wurden.
Das Ziel ist mit Unterstützung von Software eine effizente Möglichkeit bereitzustellen, um die Zuverlässigkeit der\gls{gls:DeNovo} zu erhöhen. 



\section{De-Novo-Peptidsequenzierung und Spektrums-Graphen im Detail}
\subsection{Datengewinnung}
\subsection{Datenauswertung}
 

\section{Ergebnisse/Evaluierung}
\subsection{Probleme in der Praxis}
\subsection{Lösungsansätze}


\section{Zusammenfassung}
\subsection{Ungelöste Probleme}
\subsection{Kritische Betrachtung}


\section{Section}
\cite{OpenPNovo}
\cite{pNovoPlus}

\begingroup
\setlength{\emergencystretch}{.5em}
\printbibliography
\endgroup

\appendix
\section{Appendix Section}
Appendix



\end{document}
%%%%% %%%%% %%%%% %%%%% %%%%% \end{document} %%%%% %%%%% %%%%% %%%%% %%%%%
